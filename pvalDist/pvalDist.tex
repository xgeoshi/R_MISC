\documentclass[]{article}
\usepackage{lmodern}
\usepackage{amssymb,amsmath}
\usepackage{ifxetex,ifluatex}
\usepackage{fixltx2e} % provides \textsubscript
\ifnum 0\ifxetex 1\fi\ifluatex 1\fi=0 % if pdftex
  \usepackage[T1]{fontenc}
  \usepackage[utf8]{inputenc}
\else % if luatex or xelatex
  \ifxetex
    \usepackage{mathspec}
  \else
    \usepackage{fontspec}
  \fi
  \defaultfontfeatures{Ligatures=TeX,Scale=MatchLowercase}
\fi
% use upquote if available, for straight quotes in verbatim environments
\IfFileExists{upquote.sty}{\usepackage{upquote}}{}
% use microtype if available
\IfFileExists{microtype.sty}{%
\usepackage{microtype}
\UseMicrotypeSet[protrusion]{basicmath} % disable protrusion for tt fonts
}{}
\usepackage[margin=1in]{geometry}
\usepackage{hyperref}
\hypersetup{unicode=true,
            pdftitle={Simulation of Revenue Distribution in R language. Process Reveal.},
            pdfborder={0 0 0},
            breaklinks=true}
\urlstyle{same}  % don't use monospace font for urls
\usepackage{color}
\usepackage{fancyvrb}
\newcommand{\VerbBar}{|}
\newcommand{\VERB}{\Verb[commandchars=\\\{\}]}
\DefineVerbatimEnvironment{Highlighting}{Verbatim}{commandchars=\\\{\}}
% Add ',fontsize=\small' for more characters per line
\usepackage{framed}
\definecolor{shadecolor}{RGB}{248,248,248}
\newenvironment{Shaded}{\begin{snugshade}}{\end{snugshade}}
\newcommand{\AlertTok}[1]{\textcolor[rgb]{0.94,0.16,0.16}{#1}}
\newcommand{\AnnotationTok}[1]{\textcolor[rgb]{0.56,0.35,0.01}{\textbf{\textit{#1}}}}
\newcommand{\AttributeTok}[1]{\textcolor[rgb]{0.77,0.63,0.00}{#1}}
\newcommand{\BaseNTok}[1]{\textcolor[rgb]{0.00,0.00,0.81}{#1}}
\newcommand{\BuiltInTok}[1]{#1}
\newcommand{\CharTok}[1]{\textcolor[rgb]{0.31,0.60,0.02}{#1}}
\newcommand{\CommentTok}[1]{\textcolor[rgb]{0.56,0.35,0.01}{\textit{#1}}}
\newcommand{\CommentVarTok}[1]{\textcolor[rgb]{0.56,0.35,0.01}{\textbf{\textit{#1}}}}
\newcommand{\ConstantTok}[1]{\textcolor[rgb]{0.00,0.00,0.00}{#1}}
\newcommand{\ControlFlowTok}[1]{\textcolor[rgb]{0.13,0.29,0.53}{\textbf{#1}}}
\newcommand{\DataTypeTok}[1]{\textcolor[rgb]{0.13,0.29,0.53}{#1}}
\newcommand{\DecValTok}[1]{\textcolor[rgb]{0.00,0.00,0.81}{#1}}
\newcommand{\DocumentationTok}[1]{\textcolor[rgb]{0.56,0.35,0.01}{\textbf{\textit{#1}}}}
\newcommand{\ErrorTok}[1]{\textcolor[rgb]{0.64,0.00,0.00}{\textbf{#1}}}
\newcommand{\ExtensionTok}[1]{#1}
\newcommand{\FloatTok}[1]{\textcolor[rgb]{0.00,0.00,0.81}{#1}}
\newcommand{\FunctionTok}[1]{\textcolor[rgb]{0.00,0.00,0.00}{#1}}
\newcommand{\ImportTok}[1]{#1}
\newcommand{\InformationTok}[1]{\textcolor[rgb]{0.56,0.35,0.01}{\textbf{\textit{#1}}}}
\newcommand{\KeywordTok}[1]{\textcolor[rgb]{0.13,0.29,0.53}{\textbf{#1}}}
\newcommand{\NormalTok}[1]{#1}
\newcommand{\OperatorTok}[1]{\textcolor[rgb]{0.81,0.36,0.00}{\textbf{#1}}}
\newcommand{\OtherTok}[1]{\textcolor[rgb]{0.56,0.35,0.01}{#1}}
\newcommand{\PreprocessorTok}[1]{\textcolor[rgb]{0.56,0.35,0.01}{\textit{#1}}}
\newcommand{\RegionMarkerTok}[1]{#1}
\newcommand{\SpecialCharTok}[1]{\textcolor[rgb]{0.00,0.00,0.00}{#1}}
\newcommand{\SpecialStringTok}[1]{\textcolor[rgb]{0.31,0.60,0.02}{#1}}
\newcommand{\StringTok}[1]{\textcolor[rgb]{0.31,0.60,0.02}{#1}}
\newcommand{\VariableTok}[1]{\textcolor[rgb]{0.00,0.00,0.00}{#1}}
\newcommand{\VerbatimStringTok}[1]{\textcolor[rgb]{0.31,0.60,0.02}{#1}}
\newcommand{\WarningTok}[1]{\textcolor[rgb]{0.56,0.35,0.01}{\textbf{\textit{#1}}}}
\usepackage{longtable,booktabs}
\usepackage{graphicx,grffile}
\makeatletter
\def\maxwidth{\ifdim\Gin@nat@width>\linewidth\linewidth\else\Gin@nat@width\fi}
\def\maxheight{\ifdim\Gin@nat@height>\textheight\textheight\else\Gin@nat@height\fi}
\makeatother
% Scale images if necessary, so that they will not overflow the page
% margins by default, and it is still possible to overwrite the defaults
% using explicit options in \includegraphics[width, height, ...]{}
\setkeys{Gin}{width=\maxwidth,height=\maxheight,keepaspectratio}
\IfFileExists{parskip.sty}{%
\usepackage{parskip}
}{% else
\setlength{\parindent}{0pt}
\setlength{\parskip}{6pt plus 2pt minus 1pt}
}
\setlength{\emergencystretch}{3em}  % prevent overfull lines
\providecommand{\tightlist}{%
  \setlength{\itemsep}{0pt}\setlength{\parskip}{0pt}}
\setcounter{secnumdepth}{0}
% Redefines (sub)paragraphs to behave more like sections
\ifx\paragraph\undefined\else
\let\oldparagraph\paragraph
\renewcommand{\paragraph}[1]{\oldparagraph{#1}\mbox{}}
\fi
\ifx\subparagraph\undefined\else
\let\oldsubparagraph\subparagraph
\renewcommand{\subparagraph}[1]{\oldsubparagraph{#1}\mbox{}}
\fi

%%% Use protect on footnotes to avoid problems with footnotes in titles
\let\rmarkdownfootnote\footnote%
\def\footnote{\protect\rmarkdownfootnote}

%%% Change title format to be more compact
\usepackage{titling}

% Create subtitle command for use in maketitle
\providecommand{\subtitle}[1]{
  \posttitle{
    \begin{center}\large#1\end{center}
    }
}

\setlength{\droptitle}{-2em}

  \title{Simulation of Revenue Distribution in R language. Process Reveal.}
    \pretitle{\vspace{\droptitle}\centering\huge}
  \posttitle{\par}
    \author{}
    \preauthor{}\postauthor{}
    \date{}
    \predate{}\postdate{}
  
\usepackage{booktabs}
\usepackage{longtable}
\usepackage{array}
\usepackage{multirow}
\usepackage{wrapfig}
\usepackage{float}
\usepackage{colortbl}
\usepackage{pdflscape}
\usepackage{tabu}
\usepackage{threeparttable}
\usepackage{threeparttablex}
\usepackage[normalem]{ulem}
\usepackage{makecell}
\usepackage{xcolor}

\begin{document}
\maketitle

{
\setcounter{tocdepth}{4}
\tableofcontents
}
~

\hypertarget{intro.}{%
\subsection{Intro.}\label{intro.}}

Description of the algorithm simulating total Revenue (Revenue
Distribution) upon given Opportunities, their amounts and the prediction
of the probability of winning.

In the heart of the simulation is the sum of outcomes of Bernoulli
trials (or binomial trial) which is a random experiment with exactly two
possible outcomes per Opportunity case, ``success'' or ``failure'',
where the probability of success is the near the same (every time the
experiment is conducted within hundreds thousand iteration.

\hypertarget{generate-small-random-dataset-to-start-with.}{%
\subsection{Generate small random dataset to start
with.}\label{generate-small-random-dataset-to-start-with.}}

\hypertarget{why-using-small-dataset}{%
\subsubsection{Why using small dataset}\label{why-using-small-dataset}}

In order to make it easier to understand how the full cycle of
simulation works and what results it leads to, I apply it to
artificially created small dataset:

\hypertarget{data-exploration}{%
\subsubsection{Data exploration}\label{data-exploration}}

Simulated data source have size of 3 columns \& 9 rows.

3 columns:

\begin{itemize}
\tightlist
\item
  \texttt{case\ id} - simulated case id number
\item
  \texttt{prob} - success probability of e.g.~Opportunity
\item
  \texttt{revenue} - revenue amount per Opportunity
\end{itemize}

9 rows:

\begin{itemize}
\tightlist
\item
  9 cases
\end{itemize}

Let's have a look on a small dataset in table format:

\begin{longtable}[]{@{}lrr@{}}
\toprule
case & prob & revenue\tabularnewline
\midrule
\endhead
case1 & 0.8500 & 15000\tabularnewline
case2 & 0.2345 & 10000\tabularnewline
case3 & 0.0555 & 5000\tabularnewline
case4 & 0.0010 & 5000\tabularnewline
case5 & 0.3500 & 7000\tabularnewline
case6 & 0.1600 & 2000\tabularnewline
case7 & 0.6800 & 3000\tabularnewline
case8 & 0.4000 & 4000\tabularnewline
case9 & 0.1200 & 1000\tabularnewline
\bottomrule
\end{longtable}

\hypertarget{pipeline}{%
\subsubsection{Pipeline:}\label{pipeline}}

In order to create revenue distribution (simulations) we will go through
the following steps:

\begin{enumerate}
\def\labelenumi{\alph{enumi})}
\tightlist
\item
  Generate Probability Deviation Distribution for each of 9 case's
  probability values
\item
  Simulate binary outcomes per case (success-failure) based on randomly
  chosen single prob value per distribution from part a.
\item
  Sum binary outcomes multiplied by related contract Value. x 1Mln =
  Simulated Revenue Distribution
\end{enumerate}

~

\hypertarget{pipeline-part-1.-generate-probability-deviation-distribution}{%
\paragraph{Pipeline: part 1. Generate Probability Deviation
Distribution}\label{pipeline-part-1.-generate-probability-deviation-distribution}}

Probability Deviation Distributions computed as a vector of certain
length of zeroes and ones (\texttt{"Bernulli\ trials"}) with given
probability of success. This is required to simulate variablity in
probability prediction due to natural random chance.

Let's have a look on a vector, say length = 25 with given 30\%
probability of success.

~

Function Evaluation Output:

\begin{Shaded}
\begin{Highlighting}[]
\KeywordTok{set.seed}\NormalTok{(}\DecValTok{22}\NormalTok{)}
\KeywordTok{print}\NormalTok{(bernulli_trial_}\DecValTok{1}\NormalTok{ <-}\StringTok{ }\KeywordTok{bernulli}\NormalTok{(}\DataTypeTok{success_p =} \FloatTok{0.30}\NormalTok{, }\DataTypeTok{length =} \DecValTok{25}\NormalTok{))}
\end{Highlighting}
\end{Shaded}

\begin{verbatim}
##  [1] 0 0 1 0 1 1 0 1 0 0 1 0 0 1 0 0 0 0 0 0 0 0 0 1 1
\end{verbatim}

~

As we can see vector's mean value is near given 0.3 probability since
it's randomly generated:

\begin{Shaded}
\begin{Highlighting}[]
\KeywordTok{mean}\NormalTok{(bernulli_trial_}\DecValTok{1}\NormalTok{)}
\end{Highlighting}
\end{Shaded}

\begin{verbatim}
## [1] 0.32
\end{verbatim}

~

If we run same function again we would get slightly different result,
since function is randomized over the mean and it should follow normal
distribution if all the required conditions are met:

\begin{Shaded}
\begin{Highlighting}[]
\KeywordTok{set.seed}\NormalTok{(}\DecValTok{14}\NormalTok{)}
\KeywordTok{print}\NormalTok{(bernulli_trial_}\DecValTok{2}\NormalTok{ <-}\StringTok{ }\KeywordTok{bernulli}\NormalTok{(}\DataTypeTok{success_p =} \FloatTok{0.30}\NormalTok{, }\DataTypeTok{length =} \DecValTok{25}\NormalTok{))}
\end{Highlighting}
\end{Shaded}

\begin{verbatim}
##  [1] 0 0 1 0 1 0 1 0 0 0 1 0 0 1 1 1 0 0 1 0 0 0 0 0 1
\end{verbatim}

~

Check second vector mean value again:

\begin{Shaded}
\begin{Highlighting}[]
\KeywordTok{mean}\NormalTok{(bernulli_trial_}\DecValTok{2}\NormalTok{)}
\end{Highlighting}
\end{Shaded}

\begin{verbatim}
## [1] 0.36
\end{verbatim}

~

Repeating this process = 10K times will be sufficient for our purposes
of modeling normal distribution from the given value for each
probability value.

But before proceeding with that we need to make sure the length of the
bernulli vector is sufficient to meet
\texttt{10\ \textgreater{}\ success-failure} binomial model conditions
required for near normal distribution.

We perform a check up of the simulated vector of length 25 to be
sufficient in case of given 30\% probability success, as follows:

\begin{Shaded}
\begin{Highlighting}[]
\KeywordTok{t}\NormalTok{(}\KeywordTok{c}\NormalTok{(}\StringTok{"25 * 0.3"}\NormalTok{ =}\StringTok{ }\FloatTok{0.3} \OperatorTok{*}\StringTok{ }\DecValTok{25}\NormalTok{, }\StringTok{"25 * 0.7"}\NormalTok{ =}\StringTok{ }\NormalTok{(}\DecValTok{1} \OperatorTok{-}\StringTok{ }\FloatTok{0.3}\NormalTok{) }\OperatorTok{*}\StringTok{ }\DecValTok{25}\NormalTok{))}
\end{Highlighting}
\end{Shaded}

\begin{verbatim}
##      25 * 0.3 25 * 0.7
## [1,]      7.5     17.5
\end{verbatim}

~

7.5 is less then \textgreater{} 10 necessary condition of successes so
we need to set sufficient longer vector using the following equation:

\begin{Shaded}
\begin{Highlighting}[]
\KeywordTok{print}\NormalTok{(length_for_30p <-}\StringTok{ }\DecValTok{10} \OperatorTok{/}\StringTok{ }\KeywordTok{min}\NormalTok{(}\KeywordTok{c}\NormalTok{(}\DecValTok{1} \OperatorTok{-}\StringTok{ }\FloatTok{0.3}\NormalTok{, }\FloatTok{0.3}\NormalTok{)))}
\end{Highlighting}
\end{Shaded}

\begin{verbatim}
## [1] 33.33333
\end{verbatim}

~

Let's check the sufficiency of the new vetor length \textgreater{} 10
success condition:

\begin{Shaded}
\begin{Highlighting}[]
\KeywordTok{t}\NormalTok{(}\KeywordTok{c}\NormalTok{(}\StringTok{"0.3"}\NormalTok{ =}\StringTok{ }\NormalTok{length_for_30p }\OperatorTok{*}\StringTok{ }\FloatTok{0.3}\NormalTok{, }\StringTok{"0.7"}\NormalTok{ =}\StringTok{ }\NormalTok{length_for_30p }\OperatorTok{*}\StringTok{ }\NormalTok{(}\DecValTok{1} \OperatorTok{-}\StringTok{ }\FloatTok{0.3}\NormalTok{)))}
\end{Highlighting}
\end{Shaded}

\begin{verbatim}
##      0.3      0.7
## [1,]  10 23.33333
\end{verbatim}

~

Done. Moreover, Since It's an experiment we need to put all the
Opportunities in the identical conditions while simulation, so we would
have to take the longest suitable vector among all cases and apply it
for every case.

Identifying the longest of sufficient vectors as a proportion for length
size (to be at least 10 success \& failures):

\begin{Shaded}
\begin{Highlighting}[]
\CommentTok{# Minimum Length (for 10 Success-Failures binomial condition) ----}
\NormalTok{min.tail <-}\StringTok{ }\KeywordTok{min}\NormalTok{(p, }\DecValTok{1} \OperatorTok{-}\StringTok{ }\NormalTok{p) }\CommentTok{# min value of all range of both heads and tails}
\KeywordTok{print}\NormalTok{(min.len  <-}\StringTok{ }\KeywordTok{ceiling}\NormalTok{(}\DecValTok{10} \OperatorTok{/}\StringTok{ }\NormalTok{min.tail)) }\CommentTok{# min length to get 10 binom successes}
\end{Highlighting}
\end{Shaded}

\begin{verbatim}
## [1] 10000
\end{verbatim}

~

Below is the function that take vector of all probabilities as an
argument and define required length of the vector for probabilities
means simulation.

Code:

Check the contents of the vector \texttt{p}:

\begin{Shaded}
\begin{Highlighting}[]
\KeywordTok{print}\NormalTok{(p)}
\end{Highlighting}
\end{Shaded}

\begin{verbatim}
## [1] 0.8500 0.2345 0.0555 0.0010 0.3500 0.1600 0.6800 0.4000 0.1200
\end{verbatim}

~

Applying vector \texttt{p} as an argument to the function written above:

\begin{Shaded}
\begin{Highlighting}[]
\KeywordTok{print}\NormalTok{(min.length <-}\StringTok{ }\KeywordTok{pMinLen}\NormalTok{(}\DataTypeTok{p =}\NormalTok{ p))}
\end{Highlighting}
\end{Shaded}

\begin{verbatim}
## 0.001 minimum length for bernulli success-failure cond: 10000
\end{verbatim}

\begin{verbatim}
## [1] 10000
\end{verbatim}

Now we need to create functions that creates distribustion of length
given above for each probability we have and repeat the process given
times.\\
10K would be enough for our purposes.

~

Executing functions mentioned above to get probabilities random
deviation distributions and viewing heading of the output list of
values:

\begin{Shaded}
\begin{Highlighting}[]
\KeywordTok{require}\NormalTok{(data.table)}
\KeywordTok{require}\NormalTok{(kableExtra)}
\NormalTok{pmeandist <-}\StringTok{ }\KeywordTok{bernulliMeansDist}\NormalTok{(}\DataTypeTok{pvec =}\NormalTok{ src}\OperatorTok{$}\NormalTok{prob,}
                               \DataTypeTok{rep =} \DecValTok{10000}\NormalTok{,}
                               \DataTypeTok{each =} \OtherTok{FALSE}\NormalTok{,}
                               \DataTypeTok{export =} \StringTok{"mtcars"}\NormalTok{)}
\CommentTok{# str(pmeandist)}
\KeywordTok{head}\NormalTok{(}\KeywordTok{as.data.table}\NormalTok{(pmeandist)) }\OperatorTok\StringTok{ }
\StringTok{  }\KeywordTok{kable}\NormalTok{(}\DataTypeTok{format =} \StringTok{"latex"}\NormalTok{, }\DataTypeTok{booktabs =}\NormalTok{ T) }\OperatorTok
\StringTok{  }\KeywordTok{kable_styling}\NormalTok{(}\DataTypeTok{latex_options =} \KeywordTok{c}\NormalTok{(}\StringTok{"striped"}\NormalTok{, }\StringTok{"scale_down"}\NormalTok{))}
\end{Highlighting}
\end{Shaded}

\begin{table}[H]
\centering
\resizebox{\linewidth}{!}{
\begin{tabular}{rrrrrrrrr}
\toprule
case1\_p0.85 & case2\_p0.2345 & case3\_p0.0555 & case4\_p0.001 & case5\_p0.35 & case6\_p0.16 & case7\_p0.68 & case8\_p0.4 & case9\_p0.12\\
\midrule
\rowcolor{gray!6}  0.8492 & 0.2364 & 0.0537 & 0.0013 & 0.3482 & 0.1611 & 0.6771 & 0.3970 & 0.1250\\
0.8477 & 0.2384 & 0.0555 & 0.0013 & 0.3518 & 0.1590 & 0.6818 & 0.4008 & 0.1224\\
\rowcolor{gray!6}  0.8492 & 0.2292 & 0.0557 & 0.0007 & 0.3508 & 0.1575 & 0.6828 & 0.3949 & 0.1178\\
0.8568 & 0.2399 & 0.0562 & 0.0005 & 0.3477 & 0.1527 & 0.6761 & 0.3954 & 0.1187\\
\rowcolor{gray!6}  0.8472 & 0.2319 & 0.0587 & 0.0007 & 0.3495 & 0.1591 & 0.6744 & 0.4069 & 0.1231\\
\addlinespace
0.8519 & 0.2360 & 0.0537 & 0.0008 & 0.3533 & 0.1544 & 0.6751 & 0.3988 & 0.1195\\
\bottomrule
\end{tabular}}
\end{table}

~

Let's Plot histograms of Probability Deviations for every case we had:

\includegraphics{pvalDist_files/figure-latex/plot_p_dists-1.pdf}

\newpage

\hypertarget{pipeline-part-2.-simulate-binary-outcome-per-case}{%
\paragraph{Pipeline: part 2. Simulate binary outcome per
case}\label{pipeline-part-2.-simulate-binary-outcome-per-case}}

(success-failure outcome given success probability)

Now we need to simulate binary oucomes upon simulated probabilities
distrbutions.

~

Example of outcome after repeated running for x10 times:

\begin{Shaded}
\begin{Highlighting}[]
\KeywordTok{set.seed}\NormalTok{(}\DecValTok{2}\NormalTok{)}
\KeywordTok{as.data.table}\NormalTok{(}\KeywordTok{replicate}\NormalTok{(}\DecValTok{10}\NormalTok{, }\KeywordTok{binarySim}\NormalTok{(pmeandist))) }\OperatorTok\StringTok{ }\KeywordTok{kable}\NormalTok{()}
\end{Highlighting}
\end{Shaded}

\begin{tabular}{r|r|r|r|r|r|r|r|r|r}
\hline
V1 & V2 & V3 & V4 & V5 & V6 & V7 & V8 & V9 & V10\\
\hline
1 & 1 & 1 & 1 & 1 & 1 & 1 & 1 & 0 & 1\\
\hline
0 & 1 & 0 & 0 & 0 & 0 & 1 & 1 & 0 & 0\\
\hline
0 & 0 & 0 & 1 & 0 & 0 & 0 & 0 & 0 & 0\\
\hline
0 & 0 & 0 & 0 & 0 & 0 & 0 & 0 & 0 & 0\\
\hline
0 & 0 & 0 & 0 & 1 & 0 & 0 & 1 & 0 & 1\\
\hline
0 & 0 & 0 & 0 & 1 & 0 & 0 & 0 & 0 & 0\\
\hline
0 & 0 & 1 & 1 & 1 & 1 & 1 & 0 & 1 & 1\\
\hline
0 & 0 & 0 & 0 & 1 & 0 & 1 & 0 & 1 & 0\\
\hline
0 & 0 & 0 & 0 & 0 & 0 & 0 & 0 & 0 & 0\\
\hline
\end{tabular}

~

\hypertarget{pipeline-part-3.-binary-outcomes-multiply-by-contract-value.}{%
\paragraph{Pipeline: part 3. Binary outcomes multiply by contract
Value.}\label{pipeline-part-3.-binary-outcomes-multiply-by-contract-value.}}

Finaly we add everything up. We use 10000 simulated binary outcomes per
each case, combine it with related Opportunity ID revenue and sum all
values to have simulation of Total Revnue. And Repeat x10K Times.

\begin{Shaded}
\begin{Highlighting}[]
\CommentTok{# Apply each binary outcome to related revenue}
\NormalTok{bernulliTimesVal <-}\StringTok{ }\ControlFlowTok{function}\NormalTok{(values, distr, }\DataTypeTok{coefs =} \OtherTok{NULL}\NormalTok{) \{}

\NormalTok{        bin.sim <-}\StringTok{ }\KeywordTok{binarySim}\NormalTok{(}\DataTypeTok{distr.p =}\NormalTok{ distr)}
        
        \ControlFlowTok{if}\NormalTok{ (}\OperatorTok{!}\KeywordTok{is.null}\NormalTok{(coefs) }\OperatorTok{&&}\StringTok{ }\KeywordTok{any}\NormalTok{(bin.sim }\OperatorTok{==}\StringTok{ }\DecValTok{1}\NormalTok{)) \{}
\NormalTok{                bin.ones.index <-}\StringTok{ }\KeywordTok{which}\NormalTok{(bin.sim }\OperatorTok{==}\StringTok{ }\DecValTok{1}\NormalTok{)}
\NormalTok{                len.ones <-}\StringTok{ }\KeywordTok{length}\NormalTok{(values[bin.ones.index])}
                
\NormalTok{                smp.coef <-}\StringTok{ }\KeywordTok{sample}\NormalTok{(coefs, len.ones, }\DataTypeTok{replace =} \OtherTok{TRUE}\NormalTok{)}
\NormalTok{                v <-}\StringTok{ }\NormalTok{values[bin.ones.index]}
                \KeywordTok{stopifnot}\NormalTok{(}\KeywordTok{length}\NormalTok{(v) }\OperatorTok{==}\StringTok{ }\KeywordTok{length}\NormalTok{(smp.coef))}
\NormalTok{                values[bin.ones.index] <-}\StringTok{ }\NormalTok{v }\OperatorTok{-}\StringTok{ }\NormalTok{(v }\OperatorTok{*}\StringTok{ }\NormalTok{smp.coef)}
\NormalTok{        \}}
        
        \KeywordTok{return}\NormalTok{(}\KeywordTok{sum}\NormalTok{(values }\OperatorTok{*}\StringTok{ }\NormalTok{bin.sim))}

\NormalTok{\}}


\CommentTok{# SIMULATED TOTAL REVENUE DISTRIBUTION}
\NormalTok{pvalDist <-}\StringTok{ }\ControlFlowTok{function}\NormalTok{(pvec,}
\NormalTok{                     valvec,}
                     \DataTypeTok{rep =} \DecValTok{10000}\NormalTok{,}
\NormalTok{                     parallel,}
                     \DataTypeTok{export =} \KeywordTok{c}\NormalTok{(}\StringTok{"bernulliTimesVal"}\NormalTok{, }\StringTok{"valvec"}\NormalTok{,}
                                \StringTok{"pmeandist"}\NormalTok{, }\StringTok{"binarySim"}\NormalTok{, }\StringTok{"actual"}\NormalTok{, }\StringTok{"coefs"}\NormalTok{),}
                     \DataTypeTok{coefs =} \OtherTok{NULL}\NormalTok{,}
                     \DataTypeTok{error =} \OtherTok{FALSE}\NormalTok{) \{}
        
\NormalTok{        repl.args <-}\StringTok{ }\KeywordTok{list}\NormalTok{(}\DataTypeTok{n =}\NormalTok{ rep)}
        
        \ControlFlowTok{if}\NormalTok{ (parallel) \{}
                \KeywordTok{require}\NormalTok{(future)}
                \KeywordTok{require}\NormalTok{(doFuture)}
                \KeywordTok{registerDoFuture}\NormalTok{()}
                \KeywordTok{plan}\NormalTok{(multisession)}
                
                \KeywordTok{source}\NormalTok{(}\StringTok{"_parallelRep.R"}\NormalTok{, }\DataTypeTok{local =} \OtherTok{TRUE}\NormalTok{)}
                \KeywordTok{message}\NormalTok{(}\StringTok{"<pvalDist> parallel"}\NormalTok{)}
\NormalTok{                repl.args[[}\StringTok{"each"}\NormalTok{]] <-}\StringTok{ }\OtherTok{TRUE}
\NormalTok{                repl.args[[}\StringTok{"export"}\NormalTok{]] <-}\StringTok{ }\NormalTok{export}
\NormalTok{                replX <-}\StringTok{ }\NormalTok{parallelRep}
                
\NormalTok{        \} }\ControlFlowTok{else}\NormalTok{ \{}
\NormalTok{                replX <-}\StringTok{ }\NormalTok{replicate}
\NormalTok{        \}}
        
        \CommentTok{# Creating list of berbulli means distribution}
\NormalTok{        pmeandist <-}\StringTok{ }\KeywordTok{bernulliMeansDist}\NormalTok{(}\DataTypeTok{pvec =}\NormalTok{ pvec, }\DataTypeTok{each =}\NormalTok{ parallel,}
                                       \DataTypeTok{export =}\NormalTok{ export, }\DataTypeTok{error =}\NormalTok{ error)}
        \KeywordTok{stopifnot}\NormalTok{(}\KeywordTok{identical}\NormalTok{(}\KeywordTok{length}\NormalTok{(pmeandist), }\KeywordTok{length}\NormalTok{(valvec), }\KeywordTok{length}\NormalTok{(pvec)))}
        
        \CommentTok{# Adding argument}
\NormalTok{        repl.args[[}\StringTok{"expr"}\NormalTok{]] <-}\StringTok{ }\KeywordTok{quote}\NormalTok{(}\KeywordTok{bernulliTimesVal}\NormalTok{(}\DataTypeTok{values =}\NormalTok{ valvec,}
                                                      \DataTypeTok{distr =}\NormalTok{ pmeandist,}
                                                      \DataTypeTok{coefs =}\NormalTok{ coefs))}
        
        \CommentTok{# revenue simulations}
\NormalTok{        rsim <-}\StringTok{ }\KeywordTok{do.call}\NormalTok{(replX, repl.args)}
        
        \KeywordTok{stopifnot}\NormalTok{(}\KeywordTok{length}\NormalTok{(rsim) }\OperatorTok{==}\StringTok{ }\NormalTok{rep)}
        \KeywordTok{return}\NormalTok{(rsim)}
\NormalTok{\}}
\end{Highlighting}
\end{Shaded}

\begin{Shaded}
\begin{Highlighting}[]
\KeywordTok{setwd}\NormalTok{(}\StringTok{"../pvalDist"}\NormalTok{)}
\KeywordTok{set.seed}\NormalTok{(}\DecValTok{1}\NormalTok{)}
\NormalTok{simdist <-}\StringTok{ }\KeywordTok{pvalDist}\NormalTok{(}\DataTypeTok{pvec     =}\NormalTok{ src}\OperatorTok{$}\NormalTok{prob,}
                    \DataTypeTok{valvec   =}\NormalTok{ src}\OperatorTok{$}\NormalTok{revenue,}
                    \DataTypeTok{rep      =} \DecValTok{10000}\NormalTok{,}
                    \DataTypeTok{parallel =} \OtherTok{TRUE}\NormalTok{,}
                    \DataTypeTok{error    =} \OtherTok{FALSE}\NormalTok{)}
\end{Highlighting}
\end{Shaded}

Output head:

\begin{Shaded}
\begin{Highlighting}[]
\KeywordTok{str}\NormalTok{(simdist)}
\end{Highlighting}
\end{Shaded}

\begin{verbatim}
##  num [1:10000] 22000 28000 24000 25000 22000 26000 22000 24000 32000 22000 ...
\end{verbatim}

~

Total Revenue Distribution Simulation Histogram:

\begin{verbatim}
## Loading required package: ggplot2
\end{verbatim}

\begin{verbatim}
## Loading required package: scales
\end{verbatim}

\begin{Shaded}
\begin{Highlighting}[]
\KeywordTok{plotdist}\NormalTok{(}\DataTypeTok{simdist =}\NormalTok{ simdist)}
\end{Highlighting}
\end{Shaded}

\includegraphics{pvalDist_files/figure-latex/plotartif-1.pdf}

\hypertarget{conclusion.}{%
\subsubsection{Conclusion.}\label{conclusion.}}

Mathematical Expectation is very near to the mean of simulated
distribution so we may assume that simulation works as expected. Cases
if actual value would be out of the simulated range can be explained
just by poor probabilities estimation e.g.~the closer real value to
simulated mean the better probabilities are.

Mathematical Expectation:

\begin{Shaded}
\begin{Highlighting}[]
\NormalTok{scales}\OperatorTok{::}\KeywordTok{dollar}\NormalTok{(}\KeywordTok{sum}\NormalTok{(src}\OperatorTok{$}\NormalTok{revenue }\OperatorTok{*}\StringTok{ }\NormalTok{src}\OperatorTok{$}\NormalTok{p))}
\end{Highlighting}
\end{Shaded}

\begin{verbatim}
## [1] "$21,907.50"
\end{verbatim}

\newpage

\hypertarget{real-data}{%
\subsection{Real Data}\label{real-data}}

Now we'll apply identical copy-paste approach to real data and add some
extra features on top: a) Generate Probability Deviation Distribution
for each of case's probability values b) Simulate binary outcomes per
case (success-failure) based on randomly chosen single prob value per
distribution from part a. c) Sum binary outcomes multiplied by related
contract Value and changes coefficient. x 1Mln = Simulated

We'll use Opportunities snapshot as of 2018 October 1 and filter those
are closed (won or lost) as for now. Thus we know the outcome in advance
for evaluating simulation accuracy.

\scriptsize

SQL Code for source creation:

\begin{Shaded}
\begin{Highlighting}[]
\KeywordTok{require}\NormalTok{(data.table)}
\KeywordTok{require}\NormalTok{(dplyr)}

\CommentTok{# source("../../R_OPPO_PROB/_oppoJoinProbs.R")}
\KeywordTok{source}\NormalTok{(}\StringTok{"../../R_OPPO_PROB/_probGiveInLast.R"}\NormalTok{)}
\KeywordTok{setwd}\NormalTok{(}\StringTok{"../../R_OPPO_PROB"}\NormalTok{)}
\NormalTok{data20181001 <-}\StringTok{ }\KeywordTok{setDT}\NormalTok{(}\KeywordTok{probGiveInLast}\NormalTok{(}\DataTypeTok{given_date =} \StringTok{"2018-10-01"}\NormalTok{))}
\NormalTok{actual <-}\StringTok{ }\KeywordTok{copy}\NormalTok{(data20181001)}
\end{Highlighting}
\end{Shaded}

\normalsize

~

\hypertarget{probabilities-types-to-evaluate} - project managmet expectations\\
- \texttt{Winning\_Probability} - SalesForce prediction\\
- \texttt{Predicted\ Probability} is a \texttt{Machine\ Learning}
(``hand made'') probabilities

Let's take a look on each of them:

~

\hypertarget{probability-type-1-probability}}{Probability Type 1: Probability \%}}\label{probability-type-1-probability}}

Summary statistics on \texttt{Probability\ \%} variable:

\begin{Shaded}
\begin{Highlighting}[]
\KeywordTok{require}\NormalTok{(skimr)}
\NormalTok{prp <-}\StringTok{ }\NormalTok{actual}\OperatorTok{$}\StringTok{`}\DataTypeTok{Probability %}\StringTok{`}
\KeywordTok{skim}\NormalTok{(prp) }\OperatorTok\StringTok{ }\KeywordTok{pander}\NormalTok{()}
\end{Highlighting}
\end{Shaded}

Skim summary statistics\\
n obs:\\
n variables:

\begin{longtable}[]{@{}ccccccccccc@{}}
\toprule
\begin{minipage}[b]{0.10\columnwidth}\centering
variable\strut
\end{minipage} & \begin{minipage}[b]{0.09\columnwidth}\centering
missing\strut
\end{minipage} & \begin{minipage}[b]{0.10\columnwidth}\centering
complete\strut
\end{minipage} & \begin{minipage}[b]{0.05\columnwidth}\centering
n\strut
\end{minipage} & \begin{minipage}[b]{0.06\columnwidth}\centering
mean\strut
\end{minipage} & \begin{minipage}[b]{0.06\columnwidth}\centering
sd\strut
\end{minipage} & \begin{minipage}[b]{0.04\columnwidth}\centering
p0\strut
\end{minipage} & \begin{minipage}[b]{0.05\columnwidth}\centering
p25\strut
\end{minipage} & \begin{minipage}[b]{0.05\columnwidth}\centering
p50\strut
\end{minipage} & \begin{minipage}[b]{0.05\columnwidth}\centering
p75\strut
\end{minipage} & \begin{minipage}[b]{0.06\columnwidth}\centering
p100\strut
\end{minipage}\tabularnewline
\midrule
\endhead
\begin{minipage}[t]{0.10\columnwidth}\centering
prp\strut
\end{minipage} & \begin{minipage}[t]{0.09\columnwidth}\centering
0\strut
\end{minipage} & \begin{minipage}[t]{0.10\columnwidth}\centering
988\strut
\end{minipage} & \begin{minipage}[t]{0.05\columnwidth}\centering
988\strut
\end{minipage} & \begin{minipage}[t]{0.06\columnwidth}\centering
0.27\strut
\end{minipage} & \begin{minipage}[t]{0.06\columnwidth}\centering
0.29\strut
\end{minipage} & \begin{minipage}[t]{0.04\columnwidth}\centering
0\strut
\end{minipage} & \begin{minipage}[t]{0.05\columnwidth}\centering
0\strut
\end{minipage} & \begin{minipage}[t]{0.05\columnwidth}\centering
0.1\strut
\end{minipage} & \begin{minipage}[t]{0.05\columnwidth}\centering
0.5\strut
\end{minipage} & \begin{minipage}[t]{0.06\columnwidth}\centering
0.99\strut
\end{minipage}\tabularnewline
\bottomrule
\end{longtable}

Simulating revenue distribution using \texttt{Probability\ \%} variable

\begin{Shaded}
\begin{Highlighting}[]
\NormalTok{err <-}\StringTok{ }\NormalTok{actual}\OperatorTok{$}\StringTok{`}\DataTypeTok{Probability %}\StringTok{`} \OperatorTok{-}\StringTok{ }\NormalTok{actual}\OperatorTok{$}\NormalTok{now_IsWon}

\NormalTok{sim_pp <-}\StringTok{ }\KeywordTok{pvalDist}\NormalTok{(}\DataTypeTok{pvec     =}\NormalTok{ actual}\OperatorTok{$}\StringTok{`}\DataTypeTok{Probability %}\StringTok{`}\NormalTok{,}
                   \DataTypeTok{valvec   =}\NormalTok{ actual}\OperatorTok{$}\StringTok{`}\DataTypeTok{Amount USD}\StringTok{`}\NormalTok{,}
                   \DataTypeTok{rep      =} \DecValTok{10000}\NormalTok{,}
                   \DataTypeTok{parallel =} \OtherTok{TRUE}\NormalTok{,}
                   \DataTypeTok{error    =} \OtherTok{FALSE}\NormalTok{)}
\end{Highlighting}
\end{Shaded}

\begin{verbatim}
## <pvalDist> parallel
\end{verbatim}

\begin{verbatim}
## <pMinLen>: 280 zeroes & ones are removed / total 988 p
\end{verbatim}

\begin{verbatim}
## 0.01 minimum length for bernulli success-failure cond: 1000
\end{verbatim}

\begin{Shaded}
\begin{Highlighting}[]
\NormalTok{sim_p2 <-}\StringTok{ }\KeywordTok{pvalDist}\NormalTok{(}\DataTypeTok{pvec     =}\NormalTok{ actual}\OperatorTok{$}\StringTok{`}\DataTypeTok{Probability %}\StringTok{`}\NormalTok{,}
                   \DataTypeTok{valvec   =}\NormalTok{ actual}\OperatorTok{$}\StringTok{`}\DataTypeTok{Amount USD}\StringTok{`}\NormalTok{,}
                   \DataTypeTok{rep      =} \DecValTok{10000}\NormalTok{,}
                   \DataTypeTok{parallel =} \OtherTok{TRUE}\NormalTok{,}
                   \DataTypeTok{error    =} \OtherTok{TRUE}\NormalTok{)}
\end{Highlighting}
\end{Shaded}

\begin{verbatim}
## <pvalDist> parallel
\end{verbatim}

\begin{verbatim}
## <pMinLen>: 280 zeroes & ones are removed / total 988 p
\end{verbatim}

\begin{verbatim}
## 0.01 minimum length for bernulli success-failure cond: 1000
\end{verbatim}

\begin{verbatim}
## TRUE
\end{verbatim}

Plot Revenue Distrbution upon \texttt{Probability\ \%} variable:

\begin{Shaded}
\begin{Highlighting}[]
\KeywordTok{plotdist}\NormalTok{(sim_pp)}
\end{Highlighting}
\end{Shaded}

\includegraphics{pvalDist_files/figure-latex/unnamed-chunk-14-1.pdf}

\begin{Shaded}
\begin{Highlighting}[]
\KeywordTok{plotdist}\NormalTok{(sim_p2)}
\end{Highlighting}
\end{Shaded}

\includegraphics{pvalDist_files/figure-latex/unnamed-chunk-14-2.pdf}

~

Summary statiscts of revenue distribution using
\texttt{Probability\ \%}:

\begin{Shaded}
\begin{Highlighting}[]
\KeywordTok{skim}\NormalTok{(sim_pp) }\OperatorTok\StringTok{ }\KeywordTok{pander}\NormalTok{() }\CommentTok{# all simulations statistics}
\end{Highlighting}
\end{Shaded}

Skim summary statistics\\
n obs:\\
n variables:

\begin{longtable}[]{@{}cccccccc@{}}
\caption{Table continues below}\tabularnewline
\toprule
\begin{minipage}[b]{0.11\columnwidth}\centering
variable\strut
\end{minipage} & \begin{minipage}[b]{0.10\columnwidth}\centering
missing\strut
\end{minipage} & \begin{minipage}[b]{0.11\columnwidth}\centering
complete\strut
\end{minipage} & \begin{minipage}[b]{0.08\columnwidth}\centering
n\strut
\end{minipage} & \begin{minipage}[b]{0.10\columnwidth}\centering
mean\strut
\end{minipage} & \begin{minipage}[b]{0.10\columnwidth}\centering
sd\strut
\end{minipage} & \begin{minipage}[b]{0.10\columnwidth}\centering
p0\strut
\end{minipage} & \begin{minipage}[b]{0.10\columnwidth}\centering
p25\strut
\end{minipage}\tabularnewline
\midrule
\endfirsthead
\toprule
\begin{minipage}[b]{0.11\columnwidth}\centering
variable\strut
\end{minipage} & \begin{minipage}[b]{0.10\columnwidth}\centering
missing\strut
\end{minipage} & \begin{minipage}[b]{0.11\columnwidth}\centering
complete\strut
\end{minipage} & \begin{minipage}[b]{0.08\columnwidth}\centering
n\strut
\end{minipage} & \begin{minipage}[b]{0.10\columnwidth}\centering
mean\strut
\end{minipage} & \begin{minipage}[b]{0.10\columnwidth}\centering
sd\strut
\end{minipage} & \begin{minipage}[b]{0.10\columnwidth}\centering
p0\strut
\end{minipage} & \begin{minipage}[b]{0.10\columnwidth}\centering
p25\strut
\end{minipage}\tabularnewline
\midrule
\endhead
\begin{minipage}[t]{0.11\columnwidth}\centering
sim\_pp\strut
\end{minipage} & \begin{minipage}[t]{0.10\columnwidth}\centering
0\strut
\end{minipage} & \begin{minipage}[t]{0.11\columnwidth}\centering
10000\strut
\end{minipage} & \begin{minipage}[t]{0.08\columnwidth}\centering
10000\strut
\end{minipage} & \begin{minipage}[t]{0.10\columnwidth}\centering
3.2e+08\strut
\end{minipage} & \begin{minipage}[t]{0.10\columnwidth}\centering
7.6e+07\strut
\end{minipage} & \begin{minipage}[t]{0.10\columnwidth}\centering
1.5e+08\strut
\end{minipage} & \begin{minipage}[t]{0.10\columnwidth}\centering
2.6e+08\strut
\end{minipage}\tabularnewline
\bottomrule
\end{longtable}

\begin{longtable}[]{@{}ccc@{}}
\toprule
\begin{minipage}[b]{0.13\columnwidth}\centering
p50\strut
\end{minipage} & \begin{minipage}[b]{0.13\columnwidth}\centering
p75\strut
\end{minipage} & \begin{minipage}[b]{0.13\columnwidth}\centering
p100\strut
\end{minipage}\tabularnewline
\midrule
\endhead
\begin{minipage}[t]{0.13\columnwidth}\centering
3.2e+08\strut
\end{minipage} & \begin{minipage}[t]{0.13\columnwidth}\centering
3.9e+08\strut
\end{minipage} & \begin{minipage}[t]{0.13\columnwidth}\centering
4.9e+08\strut
\end{minipage}\tabularnewline
\bottomrule
\end{longtable}

~

\hypertarget{probability-type-2-winning_probability}{%
\paragraph{\texorpdfstring{Probability Type 2:
\texttt{Winning\_Probability}}{Probability Type 2: Winning\_Probability}}\label{probability-type-2-winning_probability}}

now let's try to simulate revenues using \texttt{Winning\_Probability}
variable.

~

Summary Statistics of \texttt{Winning\_Probability} variable (423
missing values):

\begin{Shaded}
\begin{Highlighting}[]
\NormalTok{wpr <-}\StringTok{ }\NormalTok{actual}\OperatorTok{$}\NormalTok{Winning_Probability}
\KeywordTok{skim}\NormalTok{(wpr) }\OperatorTok\StringTok{ }\KeywordTok{pander}\NormalTok{()}
\end{Highlighting}
\end{Shaded}

Skim summary statistics\\
n obs:\\
n variables:

\begin{longtable}[]{@{}ccccccccccc@{}}
\toprule
\begin{minipage}[b]{0.10\columnwidth}\centering
variable\strut
\end{minipage} & \begin{minipage}[b]{0.09\columnwidth}\centering
missing\strut
\end{minipage} & \begin{minipage}[b]{0.10\columnwidth}\centering
complete\strut
\end{minipage} & \begin{minipage}[b]{0.05\columnwidth}\centering
n\strut
\end{minipage} & \begin{minipage}[b]{0.06\columnwidth}\centering
mean\strut
\end{minipage} & \begin{minipage}[b]{0.06\columnwidth}\centering
sd\strut
\end{minipage} & \begin{minipage}[b]{0.04\columnwidth}\centering
p0\strut
\end{minipage} & \begin{minipage}[b]{0.05\columnwidth}\centering
p25\strut
\end{minipage} & \begin{minipage}[b]{0.05\columnwidth}\centering
p50\strut
\end{minipage} & \begin{minipage}[b]{0.05\columnwidth}\centering
p75\strut
\end{minipage} & \begin{minipage}[b]{0.06\columnwidth}\centering
p100\strut
\end{minipage}\tabularnewline
\midrule
\endhead
\begin{minipage}[t]{0.10\columnwidth}\centering
wpr\strut
\end{minipage} & \begin{minipage}[t]{0.09\columnwidth}\centering
0\strut
\end{minipage} & \begin{minipage}[t]{0.10\columnwidth}\centering
988\strut
\end{minipage} & \begin{minipage}[t]{0.05\columnwidth}\centering
988\strut
\end{minipage} & \begin{minipage}[t]{0.06\columnwidth}\centering
0.3\strut
\end{minipage} & \begin{minipage}[t]{0.06\columnwidth}\centering
0.28\strut
\end{minipage} & \begin{minipage}[t]{0.04\columnwidth}\centering
0\strut
\end{minipage} & \begin{minipage}[t]{0.05\columnwidth}\centering
0.1\strut
\end{minipage} & \begin{minipage}[t]{0.05\columnwidth}\centering
0.2\strut
\end{minipage} & \begin{minipage}[t]{0.05\columnwidth}\centering
0.5\strut
\end{minipage} & \begin{minipage}[t]{0.06\columnwidth}\centering
1\strut
\end{minipage}\tabularnewline
\bottomrule
\end{longtable}

~

Imputation. Since there lot's of missing values we'll replace missing
\texttt{Winning\_Probability} with \texttt{Probability\ \%}

\begin{Shaded}
\begin{Highlighting}[]
\KeywordTok{require}\NormalTok{(dplyr)}
\NormalTok{ful.win.prob <-}\StringTok{ }\KeywordTok{case_when}\NormalTok{(}\KeywordTok{is.na}\NormalTok{(actual}\OperatorTok{$}\NormalTok{Winning_Probability) }\OperatorTok{~}\StringTok{ }\NormalTok{actual}\OperatorTok{$}\StringTok{`}\DataTypeTok{Probability %}\StringTok{`}\NormalTok{,}
                          \OtherTok{TRUE} \OperatorTok{~}\StringTok{ }\NormalTok{actual}\OperatorTok{$}\NormalTok{Winning_Probabilit)}
\end{Highlighting}
\end{Shaded}

\begin{Shaded}
\begin{Highlighting}[]
\KeywordTok{skim}\NormalTok{(ful.win.prob) }\OperatorTok\StringTok{ }\KeywordTok{pander}\NormalTok{() }\CommentTok{# inspecting for non missing probabilities vector}
\end{Highlighting}
\end{Shaded}

Skim summary statistics\\
n obs:\\
n variables:

\begin{longtable}[]{@{}ccccccccccc@{}}
\toprule
\begin{minipage}[b]{0.12\columnwidth}\centering
variable\strut
\end{minipage} & \begin{minipage}[b]{0.08\columnwidth}\centering
missing\strut
\end{minipage} & \begin{minipage}[b]{0.09\columnwidth}\centering
complete\strut
\end{minipage} & \begin{minipage}[b]{0.05\columnwidth}\centering
n\strut
\end{minipage} & \begin{minipage}[b]{0.06\columnwidth}\centering
mean\strut
\end{minipage} & \begin{minipage}[b]{0.06\columnwidth}\centering
sd\strut
\end{minipage} & \begin{minipage}[b]{0.04\columnwidth}\centering
p0\strut
\end{minipage} & \begin{minipage}[b]{0.05\columnwidth}\centering
p25\strut
\end{minipage} & \begin{minipage}[b]{0.05\columnwidth}\centering
p50\strut
\end{minipage} & \begin{minipage}[b]{0.05\columnwidth}\centering
p75\strut
\end{minipage} & \begin{minipage}[b]{0.06\columnwidth}\centering
p100\strut
\end{minipage}\tabularnewline
\midrule
\endhead
\begin{minipage}[t]{0.12\columnwidth}\centering
ful.win.prob\strut
\end{minipage} & \begin{minipage}[t]{0.08\columnwidth}\centering
0\strut
\end{minipage} & \begin{minipage}[t]{0.09\columnwidth}\centering
988\strut
\end{minipage} & \begin{minipage}[t]{0.05\columnwidth}\centering
988\strut
\end{minipage} & \begin{minipage}[t]{0.06\columnwidth}\centering
0.3\strut
\end{minipage} & \begin{minipage}[t]{0.06\columnwidth}\centering
0.28\strut
\end{minipage} & \begin{minipage}[t]{0.04\columnwidth}\centering
0\strut
\end{minipage} & \begin{minipage}[t]{0.05\columnwidth}\centering
0.1\strut
\end{minipage} & \begin{minipage}[t]{0.05\columnwidth}\centering
0.2\strut
\end{minipage} & \begin{minipage}[t]{0.05\columnwidth}\centering
0.5\strut
\end{minipage} & \begin{minipage}[t]{0.06\columnwidth}\centering
1\strut
\end{minipage}\tabularnewline
\bottomrule
\end{longtable}

~

Simulating revenue distribution using modified (imputation)
\texttt{Winning\_Probability} variable:

\begin{Shaded}
\begin{Highlighting}[]
\NormalTok{sim_win_prob <-}\StringTok{ }\KeywordTok{pvalDist}\NormalTok{(}\DataTypeTok{pvec     =}\NormalTok{ ful.win.prob,}
                         \DataTypeTok{valvec   =}\NormalTok{ actual}\OperatorTok{$}\StringTok{`}\DataTypeTok{Amount USD}\StringTok{`}\NormalTok{,}
                         \DataTypeTok{rep      =} \DecValTok{10000}\NormalTok{,}
                         \DataTypeTok{parallel =} \OtherTok{TRUE}\NormalTok{)}
\end{Highlighting}
\end{Shaded}

\begin{verbatim}
## Loading required package: future
\end{verbatim}

\begin{verbatim}
## Loading required package: doFuture
\end{verbatim}

\begin{verbatim}
## Loading required package: globals
\end{verbatim}

\begin{verbatim}
## Loading required package: iterators
\end{verbatim}

\begin{verbatim}
## Loading required package: parallel
\end{verbatim}

\begin{verbatim}
## <pvalDist> parallel
\end{verbatim}

\begin{verbatim}
## <pMinLen>: 218 zeroes & ones are removed / total 988 p
\end{verbatim}

\begin{verbatim}
## 0.1 minimum length for bernulli success-failure cond: 101
\end{verbatim}

Plot revenue distribution upon \texttt{Winning\_Probability} variable:

\begin{Shaded}
\begin{Highlighting}[]
\KeywordTok{plotdist}\NormalTok{(sim_win_prob)}
\end{Highlighting}
\end{Shaded}

\includegraphics{pvalDist_files/figure-latex/unnamed-chunk-20-1.pdf}

Revenue Distribution summary statistics:

\begin{Shaded}
\begin{Highlighting}[]
\KeywordTok{skim}\NormalTok{(sim_win_prob) }\OperatorTok\StringTok{ }\KeywordTok{pander}\NormalTok{() }\CommentTok{# all simulations statistics}
\end{Highlighting}
\end{Shaded}

Skim summary statistics\\
n obs:\\
n variables:

\begin{longtable}[]{@{}ccccccc@{}}
\caption{Table continues below}\tabularnewline
\toprule
\begin{minipage}[b]{0.16\columnwidth}\centering
variable\strut
\end{minipage} & \begin{minipage}[b]{0.11\columnwidth}\centering
missing\strut
\end{minipage} & \begin{minipage}[b]{0.12\columnwidth}\centering
complete\strut
\end{minipage} & \begin{minipage}[b]{0.09\columnwidth}\centering
n\strut
\end{minipage} & \begin{minipage}[b]{0.11\columnwidth}\centering
mean\strut
\end{minipage} & \begin{minipage}[b]{0.11\columnwidth}\centering
sd\strut
\end{minipage} & \begin{minipage}[b]{0.11\columnwidth}\centering
p0\strut
\end{minipage}\tabularnewline
\midrule
\endfirsthead
\toprule
\begin{minipage}[b]{0.16\columnwidth}\centering
variable\strut
\end{minipage} & \begin{minipage}[b]{0.11\columnwidth}\centering
missing\strut
\end{minipage} & \begin{minipage}[b]{0.12\columnwidth}\centering
complete\strut
\end{minipage} & \begin{minipage}[b]{0.09\columnwidth}\centering
n\strut
\end{minipage} & \begin{minipage}[b]{0.11\columnwidth}\centering
mean\strut
\end{minipage} & \begin{minipage}[b]{0.11\columnwidth}\centering
sd\strut
\end{minipage} & \begin{minipage}[b]{0.11\columnwidth}\centering
p0\strut
\end{minipage}\tabularnewline
\midrule
\endhead
\begin{minipage}[t]{0.16\columnwidth}\centering
sim\_win\_prob\strut
\end{minipage} & \begin{minipage}[t]{0.11\columnwidth}\centering
0\strut
\end{minipage} & \begin{minipage}[t]{0.12\columnwidth}\centering
10000\strut
\end{minipage} & \begin{minipage}[t]{0.09\columnwidth}\centering
10000\strut
\end{minipage} & \begin{minipage}[t]{0.11\columnwidth}\centering
3.4e+08\strut
\end{minipage} & \begin{minipage}[t]{0.11\columnwidth}\centering
7.5e+07\strut
\end{minipage} & \begin{minipage}[t]{0.11\columnwidth}\centering
1.7e+08\strut
\end{minipage}\tabularnewline
\bottomrule
\end{longtable}

\begin{longtable}[]{@{}cccc@{}}
\toprule
\begin{minipage}[b]{0.12\columnwidth}\centering
p25\strut
\end{minipage} & \begin{minipage}[b]{0.12\columnwidth}\centering
p50\strut
\end{minipage} & \begin{minipage}[b]{0.10\columnwidth}\centering
p75\strut
\end{minipage} & \begin{minipage}[b]{0.12\columnwidth}\centering
p100\strut
\end{minipage}\tabularnewline
\midrule
\endhead
\begin{minipage}[t]{0.12\columnwidth}\centering
2.6e+08\strut
\end{minipage} & \begin{minipage}[t]{0.12\columnwidth}\centering
3.4e+08\strut
\end{minipage} & \begin{minipage}[t]{0.10\columnwidth}\centering
4e+08\strut
\end{minipage} & \begin{minipage}[t]{0.12\columnwidth}\centering
5.2e+08\strut
\end{minipage}\tabularnewline
\bottomrule
\end{longtable}

~

\hypertarget{probability-type-3-predicted-probability-machine-learning}{%
\paragraph{\texorpdfstring{Probability Type 3:
\texttt{Predicted\ Probability} (Machine
Learning)}{Probability Type 3: Predicted Probability (Machine Learning)}}\label{probability-type-3-predicted-probability-machine-learning}}

Lastly we'll simulate probabilities using Machine Learning like if we
were in 1 October 2018. All adjustment identical to what we have in
production:

\begin{Shaded}
\begin{Highlighting}[]
\CommentTok{# source("../../R_OPPO_PROB/_oppoJoinProbs.R", local = TRUE)}
\CommentTok{# nowwd <- getwd()}
\CommentTok{# setwd("../../R_OPPO_PROB")}
\CommentTok{# ml_prob <- oppoJoinProbs(given_date = "2018-10-01")}
\CommentTok{# setwd(nowwd)}
\NormalTok{actual[now_IsWon }\OperatorTok{==}\StringTok{ }\DecValTok{1}\NormalTok{]}\OperatorTok{$}\NormalTok{now_AmountUSD}
\end{Highlighting}
\end{Shaded}

\begin{verbatim}
##   [1]        0.00        0.00   582415.00        0.00  2700000.00
##   [6]    80000.00   103000.00    89941.54   926438.20   600000.00
##  [11]   285152.67   827928.00   653820.00  1020033.71  1836346.74
##  [16]     8500.00    20000.00   167928.09    50000.00    18789.89
##  [21]  1300000.00 16681914.61   145000.00   882352.94   320000.00
##  [26]  1237113.40   186488.55  1700000.00  2528068.00   504000.00
##  [31]   101400.00   995569.00  1180571.13   609242.65   102050.00
##  [36]   738470.00   300000.00   160660.21    67000.00   768000.00
##  [41]   135000.00   100000.00   400000.00  1332638.00   157000.00
##  [46]   250000.00   253000.00  1000000.00   327895.00    77205.88
##  [51]     2315.00   242500.00   317479.41    26470.59     3950.00
##  [56]   416541.98    27249.44    33029.78   470129.87  4971932.00
##  [61]   200000.00   200000.00   852000.00    95271.91    38016.00
##  [66]   180000.00    53435.11    74750.00  3167767.42   271000.00
##  [71]        0.00   480519.48   100000.00    50000.00    39649.10
##  [76]   184742.65   118167.94   125000.00    40000.00    45000.00
##  [81]   617977.53   803406.74   149090.00   161330.00   350000.00
##  [86]   500000.00   478080.00   121415.73  1437966.41     8000.00
##  [91]    80000.00   988764.04   166500.00   500000.00        0.00
##  [96]   278350.52    44000.00   235965.60   235000.00   157303.37
## [101]    37505.62   538050.00   150000.00   339734.83   225000.00
## [106]   494325.84    70786.52   561797.75   201000.00   165000.00
## [111]   277078.65   485393.26  6121905.62   167676.40    10000.00
## [116]   746428.00   225000.00   150000.00    56179.78   517270.99
## [121]   120000.00   195984.73    27182.02  4600000.00   327000.00
## [126]   273844.49   287206.74   700000.00   443202.00    37600.00
## [131]   140666.97     4700.00        0.00   105343.51    58701.30
## [136]   575000.00    43814.43    93916.85   328424.72   171000.00
## [141]        0.00   543248.05  2327560.00  1307776.00   438480.00
## [146]   221328.00   187920.00  1215000.00    47120.00    63000.00
## [151]   212480.00   102679.52   342428.57   116050.00   205000.00
## [156]  7500000.00   281750.00   418426.97    32224.72   133987.00
## [161]   167282.44   301354.96    15812.43  1488277.53   156000.00
## [166]   203000.00  1285000.00   578000.00   242000.00    80000.00
## [171]    50000.00   106966.29   400000.00    56179.78   136921.35
## [176]   110000.00   280000.00   100000.00    39985.00   112273.00
## [181]   210000.00   158706.00    20000.00   100000.00   513450.00
## [186]    84943.82    88700.00     6446.00   646753.25    77700.00
## [191]    13536.00    74448.00    74448.00    67201.12  1862350.00
## [196]    66238.95   562539.18   548302.87   100000.00    65779.10
## [201]   288263.36    70534.35    94411.76    50000.00   128015.27
## [206]    67251.91  1277425.00   860500.00    18000.00    51948.05
## [211]   255809.09    20493.26   407103.00   197880.00   695000.00
## [216]   280000.00   200000.00   248412.37    20000.00   166842.02
## [221]   105168.54  1038961.04   867541.29   371134.02   208000.00
## [226]   300000.00   165000.00   164535.22   180000.00    34909.15
## [231]   447000.00        0.00   174000.00    22572.00   800000.00
## [236]   600000.00   103000.00   531000.00   730337.08   158426.97
## [241]    16320.00   198900.00    50000.00  1483633.59    57801.12
## [246]        0.00   893000.00  1076411.24
\end{verbatim}

\begin{Shaded}
\begin{Highlighting}[]
\KeywordTok{head}\NormalTok{(actual[, }\KeywordTok{c}\NormalTok{(}\StringTok{"Opportunity_ID"}\NormalTok{, }\StringTok{"Predicted Probability"}\NormalTok{)])}
\end{Highlighting}
\end{Shaded}

\begin{tabular}{l|r}
\hline
Opportunity\_ID & Predicted Probability\\
\hline
0060J00000p62fgQAA & 0.0135521\\
\hline
0060J00000p67vHQAQ & 0.0082243\\
\hline
0060J00000p6cmfQAA & 0.0090760\\
\hline
0060J00000p6cXpQAI & 0.1271619\\
\hline
0060J00000p6dTUQAY & 0.0098393\\
\hline
0060J00000p6eNNQAY & 0.0555852\\
\hline
\end{tabular}

\begin{Shaded}
\begin{Highlighting}[]
\NormalTok{pprob <-}\StringTok{ }\NormalTok{actual}\OperatorTok{$}\StringTok{`}\DataTypeTok{Predicted Probability}\StringTok{`}
\end{Highlighting}
\end{Shaded}

Summary statistics on \texttt{Predicted\ Probability} (Machine Learning)
probabilities:

\begin{Shaded}
\begin{Highlighting}[]
\KeywordTok{skim}\NormalTok{(pprob) }\OperatorTok\StringTok{ }\KeywordTok{pander}\NormalTok{()}
\end{Highlighting}
\end{Shaded}

Skim summary statistics\\
n obs:\\
n variables:

\begin{longtable}[]{@{}ccccccccccc@{}}
\toprule
\begin{minipage}[b]{0.09\columnwidth}\centering
variable\strut
\end{minipage} & \begin{minipage}[b]{0.08\columnwidth}\centering
missing\strut
\end{minipage} & \begin{minipage}[b]{0.09\columnwidth}\centering
complete\strut
\end{minipage} & \begin{minipage}[b]{0.05\columnwidth}\centering
n\strut
\end{minipage} & \begin{minipage}[b]{0.06\columnwidth}\centering
mean\strut
\end{minipage} & \begin{minipage}[b]{0.06\columnwidth}\centering
sd\strut
\end{minipage} & \begin{minipage}[b]{0.04\columnwidth}\centering
p0\strut
\end{minipage} & \begin{minipage}[b]{0.07\columnwidth}\centering
p25\strut
\end{minipage} & \begin{minipage}[b]{0.07\columnwidth}\centering
p50\strut
\end{minipage} & \begin{minipage}[b]{0.06\columnwidth}\centering
p75\strut
\end{minipage} & \begin{minipage}[b]{0.06\columnwidth}\centering
p100\strut
\end{minipage}\tabularnewline
\midrule
\endhead
\begin{minipage}[t]{0.09\columnwidth}\centering
pprob\strut
\end{minipage} & \begin{minipage}[t]{0.08\columnwidth}\centering
0\strut
\end{minipage} & \begin{minipage}[t]{0.09\columnwidth}\centering
988\strut
\end{minipage} & \begin{minipage}[t]{0.05\columnwidth}\centering
988\strut
\end{minipage} & \begin{minipage}[t]{0.06\columnwidth}\centering
0.26\strut
\end{minipage} & \begin{minipage}[t]{0.06\columnwidth}\centering
0.33\strut
\end{minipage} & \begin{minipage}[t]{0.04\columnwidth}\centering
0\strut
\end{minipage} & \begin{minipage}[t]{0.07\columnwidth}\centering
0.025\strut
\end{minipage} & \begin{minipage}[t]{0.07\columnwidth}\centering
0.082\strut
\end{minipage} & \begin{minipage}[t]{0.06\columnwidth}\centering
0.39\strut
\end{minipage} & \begin{minipage}[t]{0.06\columnwidth}\centering
0.99\strut
\end{minipage}\tabularnewline
\bottomrule
\end{longtable}

~

Now we'll join predicted probabilities for given date to our
\texttt{actual} dataset:

\begin{Shaded}
\begin{Highlighting}[]
\CommentTok{# join_ml <- left_join(actual, ml_prob_slim, by = "Opportunity ID")}
\CommentTok{# head(join_ml)}
\end{Highlighting}
\end{Shaded}

~

now let's see if there are any missing..

\begin{Shaded}
\begin{Highlighting}[]
\CommentTok{# pprob.y <- join_ml$`Predicted Probability.y`}
\CommentTok{# skim(pprob.y) %>% pander()}
\end{Highlighting}
\end{Shaded}

~

Perfect! Nothing (0) is missing. So we can start simulation.

\begin{Shaded}
\begin{Highlighting}[]
\NormalTok{sim_ml_prob <-}\StringTok{ }\KeywordTok{pvalDist}\NormalTok{(}\DataTypeTok{pvec =}\NormalTok{ actual}\OperatorTok{$}\StringTok{`}\DataTypeTok{Predicted Probability}\StringTok{`}\NormalTok{,}
                         \DataTypeTok{valvec =}\NormalTok{ actual}\OperatorTok{$}\StringTok{`}\DataTypeTok{Amount USD}\StringTok{`}\NormalTok{,}
                         \DataTypeTok{rep =} \DecValTok{10000}\NormalTok{,}
                         \DataTypeTok{parallel =} \OtherTok{TRUE}\NormalTok{)}
\end{Highlighting}
\end{Shaded}

Summary statistics of simulated Revenue Distribution using
\texttt{Predicted\ Probability} (Machine Learning) probailities

\begin{Shaded}
\begin{Highlighting}[]
\KeywordTok{skim}\NormalTok{(sim_ml_prob) }\OperatorTok\StringTok{ }\KeywordTok{pander}\NormalTok{()}
\end{Highlighting}
\end{Shaded}

Skim summary statistics\\
n obs:\\
n variables:

\begin{longtable}[]{@{}ccccccc@{}}
\caption{Table continues below}\tabularnewline
\toprule
\begin{minipage}[b]{0.15\columnwidth}\centering
variable\strut
\end{minipage} & \begin{minipage}[b]{0.11\columnwidth}\centering
missing\strut
\end{minipage} & \begin{minipage}[b]{0.12\columnwidth}\centering
complete\strut
\end{minipage} & \begin{minipage}[b]{0.09\columnwidth}\centering
n\strut
\end{minipage} & \begin{minipage}[b]{0.11\columnwidth}\centering
mean\strut
\end{minipage} & \begin{minipage}[b]{0.11\columnwidth}\centering
sd\strut
\end{minipage} & \begin{minipage}[b]{0.11\columnwidth}\centering
p0\strut
\end{minipage}\tabularnewline
\midrule
\endfirsthead
\toprule
\begin{minipage}[b]{0.15\columnwidth}\centering
variable\strut
\end{minipage} & \begin{minipage}[b]{0.11\columnwidth}\centering
missing\strut
\end{minipage} & \begin{minipage}[b]{0.12\columnwidth}\centering
complete\strut
\end{minipage} & \begin{minipage}[b]{0.09\columnwidth}\centering
n\strut
\end{minipage} & \begin{minipage}[b]{0.11\columnwidth}\centering
mean\strut
\end{minipage} & \begin{minipage}[b]{0.11\columnwidth}\centering
sd\strut
\end{minipage} & \begin{minipage}[b]{0.11\columnwidth}\centering
p0\strut
\end{minipage}\tabularnewline
\midrule
\endhead
\begin{minipage}[t]{0.15\columnwidth}\centering
sim\_ml\_prob\strut
\end{minipage} & \begin{minipage}[t]{0.11\columnwidth}\centering
0\strut
\end{minipage} & \begin{minipage}[t]{0.12\columnwidth}\centering
10000\strut
\end{minipage} & \begin{minipage}[t]{0.09\columnwidth}\centering
10000\strut
\end{minipage} & \begin{minipage}[t]{0.11\columnwidth}\centering
2.2e+08\strut
\end{minipage} & \begin{minipage}[t]{0.11\columnwidth}\centering
6.3e+07\strut
\end{minipage} & \begin{minipage}[t]{0.11\columnwidth}\centering
1.2e+08\strut
\end{minipage}\tabularnewline
\bottomrule
\end{longtable}

\begin{longtable}[]{@{}cccc@{}}
\toprule
\begin{minipage}[b]{0.12\columnwidth}\centering
p25\strut
\end{minipage} & \begin{minipage}[b]{0.12\columnwidth}\centering
p50\strut
\end{minipage} & \begin{minipage}[b]{0.12\columnwidth}\centering
p75\strut
\end{minipage} & \begin{minipage}[b]{0.12\columnwidth}\centering
p100\strut
\end{minipage}\tabularnewline
\midrule
\endhead
\begin{minipage}[t]{0.12\columnwidth}\centering
1.8e+08\strut
\end{minipage} & \begin{minipage}[t]{0.12\columnwidth}\centering
1.9e+08\strut
\end{minipage} & \begin{minipage}[t]{0.12\columnwidth}\centering
2.9e+08\strut
\end{minipage} & \begin{minipage}[t]{0.12\columnwidth}\centering
4.1e+08\strut
\end{minipage}\tabularnewline
\bottomrule
\end{longtable}

Histogram of simulated Revenue Distribution using
\texttt{Predicted\ Probability} (Machine Learning) probailities

\begin{Shaded}
\begin{Highlighting}[]
\KeywordTok{plotdist}\NormalTok{(sim_ml_prob)}
\end{Highlighting}
\end{Shaded}

\includegraphics{pvalDist_files/figure-latex/unnamed-chunk-29-1.pdf}

~

\hypertarget{intermediate-conclusion.}{%
\subsubsection{Intermediate
conclusion.}\label{intermediate-conclusion.}}

We haved proved upon simulation that existing Machine Learning
Probabilities model has highest fidelity. As We can see based on 3
probabilities types, simulation mean of the one based on Machine
learning are much closer to the acutal value. Which is:

In case Contract Values per Opportinity won't change:

\begin{Shaded}
\begin{Highlighting}[]
\NormalTok{scales}\OperatorTok{::}\KeywordTok{dollar}\NormalTok{(actual[now_IsWon }\OperatorTok{==}\StringTok{ "1"}\NormalTok{, }\KeywordTok{sum}\NormalTok{(}\StringTok{`}\DataTypeTok{Amount USD}\StringTok{`}\NormalTok{)])}
\end{Highlighting}
\end{Shaded}

\begin{verbatim}
## [1] "$198,477,807"
\end{verbatim}

~

In case Contract values change during lifetime:

\begin{Shaded}
\begin{Highlighting}[]
\NormalTok{scales}\OperatorTok{::}\KeywordTok{dollar}\NormalTok{(actual[now_IsWon }\OperatorTok{==}\StringTok{ "1"}\NormalTok{, }\KeywordTok{sum}\NormalTok{(now_AmountUSD)])}
\end{Highlighting}
\end{Shaded}

\begin{verbatim}
## [1] "$127,847,316"
\end{verbatim}

\newpage

\hypertarget{adding-extra-coefficient-for-opportunity-amounts}{%
\subsubsection{Adding Extra Coefficient for Opportunity
Amounts}\label{adding-extra-coefficient-for-opportunity-amounts}}

As seen previously there are lifetime changes in the amount values by
Opportunity. Which can be seen in dataset head below. \texttt{AmountUSD}
represent amount of \texttt{Opportinity\ ID} on ``2018-10-01'',
\texttt{now\_AmountUSD} represent ``Amount USD'' but up to date:

\begin{Shaded}
\begin{Highlighting}[]
\KeywordTok{head}\NormalTok{(actual[, .(}\StringTok{`}\DataTypeTok{Opportunity_ID}\StringTok{`}\NormalTok{, }\StringTok{`}\DataTypeTok{Amount USD}\StringTok{`}\NormalTok{, }\StringTok{`}\DataTypeTok{now_AmountUSD}\StringTok{`}\NormalTok{)])}
\end{Highlighting}
\end{Shaded}

\begin{tabular}{l|r|r}
\hline
Opportunity\_ID & Amount USD & now\_AmountUSD\\
\hline
0060J00000p62fgQAA & 2418201.3 & 2247191.0\\
\hline
0060J00000p67vHQAQ & 483640.3 & 449438.2\\
\hline
0060J00000p6cmfQAA & 544095.3 & 505618.0\\
\hline
0060J00000p6cXpQAI & 100000.0 & 100000.0\\
\hline
0060J00000p6dTUQAY & 1209100.7 & 1123595.5\\
\hline
0060J00000p6eNNQAY & 1330010.7 & 1235955.1\\
\hline
\end{tabular}

~

As far as we need changes coeffients for the win cases only, we'll
download opportunities of ``2018-10-01'' from snapshot and filter those
IDs finished as win up to date:

\begin{Shaded}
\begin{Highlighting}[]
\CommentTok{# nowwd <- getwd()}
\CommentTok{# setwd("../../R_OPPO_PROB")}
\NormalTok{won <-}\StringTok{ }\NormalTok{actual[now_IsWon }\OperatorTok{==}\StringTok{ }\DecValTok{1}\NormalTok{]}
\CommentTok{# setwd(nowwd)}
\end{Highlighting}
\end{Shaded}

~

Let's see how many obseravatons (IDs) from ``2018-10-01'' meets our
criteria (won in the end):

\begin{Shaded}
\begin{Highlighting}[]
\KeywordTok{nrow}\NormalTok{(won)}
\end{Highlighting}
\end{Shaded}

\begin{verbatim}
## [1] 248
\end{verbatim}

~

For ease of \texttt{won} data exploration I'll narrow dataset keeping
only variables we are interested in: \texttt{Opportunity\ ID},
\texttt{Amount\ USD}, \texttt{now\_AmountUSD} and name new dataest as
\texttt{won.cln}. Let's view dataset head:

\begin{Shaded}
\begin{Highlighting}[]
\KeywordTok{head}\NormalTok{(won.cln <-}\StringTok{ }\NormalTok{won[, .(}\StringTok{`}\DataTypeTok{Opportunity_ID}\StringTok{`}\NormalTok{, }\StringTok{`}\DataTypeTok{Amount USD}\StringTok{`}\NormalTok{, }\StringTok{`}\DataTypeTok{now_AmountUSD}\StringTok{`}\NormalTok{)])}
\end{Highlighting}
\end{Shaded}

\begin{tabular}{l|r|r}
\hline
Opportunity\_ID & Amount USD & now\_AmountUSD\\
\hline
0060J00000p6OipQAE & 3000000 & 0\\
\hline
0060J00000p6uVqQAI & 2500000 & 0\\
\hline
0060J00000p7FWPQA2 & 400000 & 582415\\
\hline
0060J00000p7mlnQAA & 7500000 & 0\\
\hline
0060J00000p7yRbQAI & 2700000 & 2700000\\
\hline
0060J00000p8d0nQAA & 80000 & 80000\\
\hline
\end{tabular}

~

Before calculatin difference, let's investigate initial values of
\texttt{Amount\ USD} variable precisely:

\begin{Shaded}
\begin{Highlighting}[]
\KeywordTok{require}\NormalTok{(ggplot2)}
\KeywordTok{qplot}\NormalTok{(}\DataTypeTok{x =}\NormalTok{ won.cln}\OperatorTok{$}\StringTok{`}\DataTypeTok{Amount USD}\StringTok{`}\NormalTok{, }\DataTypeTok{geom =} \StringTok{"histogram"}\NormalTok{)}
\end{Highlighting}
\end{Shaded}

\includegraphics{pvalDist_files/figure-latex/unnamed-chunk-33-1.pdf}

~

\texttt{Amount\ USD} values distribution is very strongly right skewed
since there are outlier(s) with values above \$2 Mlns.\\
Also since mode value is around zero which requires detailed
investigation for errors. Let's see obseravations where
\texttt{Amount\ USD} is less then, say 5000:

\begin{Shaded}
\begin{Highlighting}[]
\NormalTok{won.cln[}\StringTok{`}\DataTypeTok{Amount USD}\StringTok{`} \OperatorTok{<}\StringTok{ }\DecValTok{5000}\NormalTok{]}
\end{Highlighting}
\end{Shaded}

\begin{tabular}{l|r|r}
\hline
Opportunity\_ID & Amount USD & now\_AmountUSD\\
\hline
0060J00000segYGQAY & 0 & 0.00\\
\hline
0060J00000tMlk8QAC & 0 & 0.00\\
\hline
0060J00000tMlkDQAS & 0 & 543248.05\\
\hline
0060J00000tNlFUQA0 & 0 & 84943.82\\
\hline
006b000000RsoFQAAZ & 0 & 0.00\\
\hline
\end{tabular}

~

There are totally 0 obseravations where \texttt{Amount\ USD} is below
5000. We can see above there one value = 1 among them which have the
same meaning as = 0 so I'll replace all 1 with 0 since it have idenical
meaning:

\begin{Shaded}
\begin{Highlighting}[]
\NormalTok{won.cln[}\StringTok{`}\DataTypeTok{Amount USD}\StringTok{`} \OperatorTok{==}\StringTok{ }\DecValTok{1}\NormalTok{, }\StringTok{`}\DataTypeTok{Amount USD}\StringTok{`} \OperatorTok{:}\ErrorTok{=}\StringTok{ }\DecValTok{0}\NormalTok{]}
\NormalTok{won.cln[}\StringTok{`}\DataTypeTok{now_AmountUSD}\StringTok{`} \OperatorTok{==}\StringTok{ }\DecValTok{1}\NormalTok{, }\StringTok{`}\DataTypeTok{now_AmountUSD}\StringTok{`} \OperatorTok{:}\ErrorTok{=}\StringTok{ }\DecValTok{0}\NormalTok{]}
\NormalTok{won.cln[}\StringTok{`}\DataTypeTok{Amount USD}\StringTok{`} \OperatorTok{<}\StringTok{ }\DecValTok{5000}\NormalTok{]}
\end{Highlighting}
\end{Shaded}

\begin{tabular}{l|r|r}
\hline
Opportunity\_ID & Amount USD & now\_AmountUSD\\
\hline
0060J00000segYGQAY & 0 & 0.00\\
\hline
0060J00000tMlk8QAC & 0 & 0.00\\
\hline
0060J00000tMlkDQAS & 0 & 543248.05\\
\hline
0060J00000tNlFUQA0 & 0 & 84943.82\\
\hline
006b000000RsoFQAAZ & 0 & 0.00\\
\hline
\end{tabular}

~

Now we are ready to apply function to calculate the percentage
difference by IDs so we could use these values as a coefficient in
distribution of the revenue simulation. Function Code:

\begin{Shaded}
\begin{Highlighting}[]
\NormalTok{diffCoef <-}\StringTok{ }\ControlFlowTok{function}\NormalTok{(begin_num,}
\NormalTok{                        compl_num,}
                        \DataTypeTok{na.rm =} \OtherTok{FALSE}\NormalTok{,}
                        \DataTypeTok{round =} \OtherTok{TRUE}\NormalTok{,}
                        \DataTypeTok{na.infinite =} \OtherTok{TRUE}\NormalTok{,}
                        \DataTypeTok{limit_quant_tails =} \DecValTok{0}\NormalTok{) \{}
    
    \KeywordTok{stopifnot}\NormalTok{(}\KeywordTok{length}\NormalTok{(begin_num) }\OperatorTok{==}\StringTok{ }\KeywordTok{length}\NormalTok{(compl_num))}
    \KeywordTok{stopifnot}\NormalTok{(}\KeywordTok{is.numeric}\NormalTok{(begin_num), }\KeywordTok{is.numeric}\NormalTok{(compl_num))}
    
\NormalTok{    begin.num <-}\StringTok{ }\NormalTok{begin_num}
\NormalTok{    compl.num <-}\StringTok{ }\NormalTok{compl_num}
    
    \CommentTok{# oppoSQL(given_date = "2018-10-01", ended_as = "all")}
\NormalTok{    coef0 <-}\StringTok{ }\NormalTok{(begin.num }\OperatorTok{==}\StringTok{ }\NormalTok{0L) }\OperatorTok{&}\StringTok{ }\NormalTok{(compl.num }\OperatorTok{==}\StringTok{ }\NormalTok{0L)}
\NormalTok{    coefs <-}\StringTok{ }\NormalTok{(begin.num }\OperatorTok{-}\StringTok{ }\NormalTok{compl.num) }\OperatorTok{/}\StringTok{ }\NormalTok{begin.num}
\NormalTok{    coefs[coef0] <-}\StringTok{ }\DecValTok{0}
    
    \ControlFlowTok{if}\NormalTok{ (na.infinite) \{}
\NormalTok{        coefs[}\KeywordTok{is.infinite}\NormalTok{(coefs)] <-}\StringTok{ }\OtherTok{NA}
\NormalTok{    \}}
    
    \ControlFlowTok{if}\NormalTok{ (round) \{}
\NormalTok{        coefs <-}\StringTok{ }\KeywordTok{round}\NormalTok{(coefs, }\DecValTok{4}\NormalTok{)}
\NormalTok{    \}}

    \ControlFlowTok{if}\NormalTok{ (}\OperatorTok{!}\KeywordTok{is.null}\NormalTok{(limit_quant_tails) }\OperatorTok{&}\StringTok{ }\KeywordTok{is.numeric}\NormalTok{(limit_quant_tails)) \{}
        \KeywordTok{stopifnot}\NormalTok{(limit_quant_tails }\OperatorTok{>=}\StringTok{ }\DecValTok{0} \OperatorTok{&}\StringTok{ }\NormalTok{limit_quant_tails }\OperatorTok{<=}\StringTok{ }\DecValTok{1}\NormalTok{)}
\NormalTok{        q.val.left <-}\StringTok{ }\KeywordTok{quantile}\NormalTok{(coefs, limit_quant_tails, }\DataTypeTok{na.rm =} \OtherTok{TRUE}\NormalTok{)}
\NormalTok{        q.val.right <-}\StringTok{ }\KeywordTok{quantile}\NormalTok{(coefs, }\DecValTok{1} \OperatorTok{-}\StringTok{ }\NormalTok{limit_quant_tails, }\DataTypeTok{na.rm =} \OtherTok{TRUE}\NormalTok{)}
        \KeywordTok{message}\NormalTok{(}\StringTok{"Limit quantile Values out of "}\NormalTok{, }\KeywordTok{names}\NormalTok{(q.val.left), }\StringTok{" "}\NormalTok{,}
                \KeywordTok{names}\NormalTok{(q.val.right), }\StringTok{": "}\NormalTok{, q.val.left, }\StringTok{" "}\NormalTok{,  q.val.right)}
\NormalTok{        coefs[coefs }\OperatorTok{<}\StringTok{ }\NormalTok{q.val.left] <-}\StringTok{ }\OtherTok{NA}
\NormalTok{        coefs[coefs }\OperatorTok{>}\StringTok{ }\NormalTok{q.val.right] <-}\StringTok{ }\OtherTok{NA}
\NormalTok{    \}}
    
    \ControlFlowTok{if}\NormalTok{ (na.rm) \{}
\NormalTok{        coefs <-}\StringTok{ }\NormalTok{coefs[}\OperatorTok{!}\KeywordTok{is.na}\NormalTok{(coefs)]}
        \KeywordTok{stopifnot}\NormalTok{(}\OperatorTok{!}\KeywordTok{is.na}\NormalTok{(coefs))}
\NormalTok{    \}}
    
    \CommentTok{#what to do with th cases start = 0, end > 0L?}
    \KeywordTok{stopifnot}\NormalTok{(}\KeywordTok{is.numeric}\NormalTok{(coefs))}
    \KeywordTok{return}\NormalTok{(coefs)}
\NormalTok{\}}
\end{Highlighting}
\end{Shaded}

~

Coefficient calculation uses following formula:
\texttt{(was\ -\ now)\ /\ was} (with 10\% percentile outliers cut from
both sides). Coefs meanings: negative coef is increase in the amount by
the end, positive is decrease (in percent).

Code:

\begin{Shaded}
\begin{Highlighting}[]
\NormalTok{won.coef <-}\StringTok{ }\NormalTok{won.cln[, diff_coef }\OperatorTok{:}\ErrorTok{=}\StringTok{ }\KeywordTok{diffCoef}\NormalTok{(}\DataTypeTok{begin_num =} \StringTok{`}\DataTypeTok{Amount USD}\StringTok{`}\NormalTok{,}
                                            \DataTypeTok{compl_num =} \StringTok{`}\DataTypeTok{now_AmountUSD}\StringTok{`}\NormalTok{,}
                                            \DataTypeTok{limit_quant_tails =} \FloatTok{0.05}\NormalTok{)]}
\NormalTok{won.coef <-}\StringTok{ }\NormalTok{won.coef[}\OperatorTok{!}\KeywordTok{is.na}\NormalTok{(diff_coef)]}
\KeywordTok{head}\NormalTok{(won.coef, }\DecValTok{20}\NormalTok{)}
\end{Highlighting}
\end{Shaded}

\begin{tabular}{l|r|r|r}
\hline
Opportunity\_ID & Amount USD & now\_AmountUSD & diff\_coef\\
\hline
0060J00000p7FWPQA2 & 400000.00 & 582415.00 & -0.4560\\
\hline
0060J00000p7yRbQAI & 2700000.00 & 2700000.00 & 0.0000\\
\hline
0060J00000p8d0nQAA & 80000.00 & 80000.00 & 0.0000\\
\hline
0060J00000pQffoQAC & 106000.00 & 103000.00 & 0.0283\\
\hline
0060J00000pUsp3QAC & 275000.00 & 89941.54 & 0.6729\\
\hline
0060J00000pVNY7QAO & 604550.33 & 926438.20 & -0.5324\\
\hline
0060J00000pW8JTQA0 & 500000.00 & 600000.00 & -0.2000\\
\hline
0060J00000qFmGiQAK & 1338582.68 & 285152.67 & 0.7870\\
\hline
0060J00000qG3TFQA0 & 737928.00 & 827928.00 & -0.1220\\
\hline
0060J00000qG3TPQA0 & 653824.00 & 653820.00 & 0.0000\\
\hline
0060J00000qG6SGQA0 & 1097657.85 & 1020033.71 & 0.0707\\
\hline
0060J00000qGE8AQAW & 2046109.51 & 1836346.74 & 0.1025\\
\hline
0060J00000qGjBxQAK & 20000.00 & 20000.00 & 0.0000\\
\hline
0060J00000qGjRvQAK & 300000.00 & 167928.09 & 0.4402\\
\hline
0060J00000qH3qfQAC & 124000.00 & 50000.00 & 0.5968\\
\hline
0060J00000qH6n8QAC & 20219.79 & 18789.89 & 0.0707\\
\hline
0060J00000qH7teQAC & 1546391.75 & 1300000.00 & 0.1593\\
\hline
0060J00000qHc74QAC & 25000148.00 & 16681914.61 & 0.3327\\
\hline
0060J00000qHgedQAC & 350000.00 & 145000.00 & 0.5857\\
\hline
0060J00000qHgiGQAS & 888888.89 & 882352.94 & 0.0074\\
\hline
\end{tabular}

\begin{Shaded}
\begin{Highlighting}[]
\KeywordTok{qplot}\NormalTok{(}\DataTypeTok{x =}\NormalTok{ won.coef}\OperatorTok{$}\NormalTok{diff_coef, }\DataTypeTok{geom =} \StringTok{"histogram"}\NormalTok{, }\DataTypeTok{binwidth =} \FloatTok{0.25}\NormalTok{,}
      \DataTypeTok{main =} \StringTok{"Coefficients Distribution"}\NormalTok{, }\DataTypeTok{xlab =} \StringTok{"coeffieinct"}\NormalTok{, }\DataTypeTok{ylab =} \StringTok{"count"}\NormalTok{)}
\end{Highlighting}
\end{Shaded}

\includegraphics{pvalDist_files/figure-latex/unnamed-chunk-36-1.pdf}

~

As we can see most of the values are whether decreased in time or stayed
without changes (mode = 0, nost of the coefs are above 0)

Now let's put these coefficients in a separate variable \texttt{coefs}

\begin{Shaded}
\begin{Highlighting}[]
\NormalTok{coefs <-}\StringTok{ }\NormalTok{won.coef}\OperatorTok{$}\NormalTok{diff_coef}
\end{Highlighting}
\end{Shaded}

~

In order to simulate changes in the amount we would randomly apply them
to those cases that appeared as won in simulation:

\begin{Shaded}
\begin{Highlighting}[]
\NormalTok{sim_pp2 <-}\StringTok{ }\KeywordTok{pvalDist}\NormalTok{(}\DataTypeTok{pvec =}\NormalTok{ actual}\OperatorTok{$}\StringTok{`}\DataTypeTok{Probability %}\StringTok{`}\NormalTok{,}
                   \DataTypeTok{valvec =}\NormalTok{ actual}\OperatorTok{$}\StringTok{`}\DataTypeTok{Amount USD}\StringTok{`}\NormalTok{,}
                   \DataTypeTok{rep =} \DecValTok{10000}\NormalTok{,}
                   \DataTypeTok{parallel =} \OtherTok{TRUE}\NormalTok{,}
                   \DataTypeTok{coefs =}\NormalTok{ coefs)}


\NormalTok{sim_win_prob2 <-}\StringTok{ }\KeywordTok{pvalDist}\NormalTok{(}\DataTypeTok{pvec =}\NormalTok{ ful.win.prob,}
                         \DataTypeTok{valvec =}\NormalTok{ actual}\OperatorTok{$}\StringTok{`}\DataTypeTok{Amount USD}\StringTok{`}\NormalTok{,}
                         \DataTypeTok{rep =} \DecValTok{10000}\NormalTok{,}
                         \DataTypeTok{parallel =} \OtherTok{TRUE}\NormalTok{,}
                         \DataTypeTok{coefs =}\NormalTok{ coefs)}


\NormalTok{sim_ml_prob2 <-}\StringTok{ }\KeywordTok{pvalDist}\NormalTok{(}\DataTypeTok{pvec =}\NormalTok{ actual}\OperatorTok{$}\StringTok{`}\DataTypeTok{Predicted Probability}\StringTok{`}\NormalTok{,}
                         \DataTypeTok{valvec =}\NormalTok{ actual}\OperatorTok{$}\StringTok{`}\DataTypeTok{Amount USD}\StringTok{`}\NormalTok{,}
                         \DataTypeTok{rep =} \DecValTok{10000}\NormalTok{,}
                         \DataTypeTok{parallel =} \OtherTok{TRUE}\NormalTok{,}
                         \DataTypeTok{coefs =}\NormalTok{ coefs)}

\NormalTok{err <-}\StringTok{ }\NormalTok{actual}\OperatorTok{$}\StringTok{`}\DataTypeTok{Predicted Probability}\StringTok{`} \OperatorTok{-}\StringTok{ }\NormalTok{actual}\OperatorTok{$}\NormalTok{now_IsWon}

\NormalTok{sim_ml_prob3 <-}\StringTok{ }\KeywordTok{pvalDist}\NormalTok{(}\DataTypeTok{pvec =}\NormalTok{ actual}\OperatorTok{$}\StringTok{`}\DataTypeTok{Predicted Probability}\StringTok{`}\NormalTok{,}
                         \DataTypeTok{valvec =}\NormalTok{ actual}\OperatorTok{$}\StringTok{`}\DataTypeTok{Amount USD}\StringTok{`}\NormalTok{,}
                         \DataTypeTok{rep =} \DecValTok{100000}\NormalTok{,}
                         \DataTypeTok{parallel =} \OtherTok{TRUE}\NormalTok{,}
                         \DataTypeTok{coefs =}\NormalTok{ coefs,}
                         \DataTypeTok{error =} \OtherTok{TRUE}\NormalTok{)}


\KeywordTok{plotdist}\NormalTok{(sim_pp2)}
\end{Highlighting}
\end{Shaded}

\includegraphics{pvalDist_files/figure-latex/allsims-1.pdf}

\begin{Shaded}
\begin{Highlighting}[]
\KeywordTok{plotdist}\NormalTok{(sim_win_prob2)}
\end{Highlighting}
\end{Shaded}

\includegraphics{pvalDist_files/figure-latex/allsims-2.pdf}

\begin{Shaded}
\begin{Highlighting}[]
\KeywordTok{plotdist}\NormalTok{(sim_ml_prob2)}
\end{Highlighting}
\end{Shaded}

\includegraphics{pvalDist_files/figure-latex/allsims-3.pdf}

\begin{Shaded}
\begin{Highlighting}[]
\NormalTok{test1 <-}\StringTok{ }\KeywordTok{data.table}\NormalTok{(}\DataTypeTok{simdist =} \KeywordTok{c}\NormalTok{(sim_pp, sim_pp2),}
                   \DataTypeTok{coef =} \KeywordTok{c}\NormalTok{(}\KeywordTok{rep}\NormalTok{(}\OtherTok{FALSE}\NormalTok{, }\KeywordTok{length}\NormalTok{(sim_pp)),}
                            \KeywordTok{rep}\NormalTok{(}\OtherTok{TRUE}\NormalTok{, }\KeywordTok{length}\NormalTok{(sim_pp2))),}
                   \DataTypeTok{Prob_Type =} \StringTok{"Probability %"}\NormalTok{)}

\NormalTok{test2 <-}\StringTok{ }\KeywordTok{data.table}\NormalTok{(}\DataTypeTok{simdist =} \KeywordTok{c}\NormalTok{(sim_win_prob, sim_win_prob2),}
                   \DataTypeTok{coef =} \KeywordTok{c}\NormalTok{(}\KeywordTok{rep}\NormalTok{(}\OtherTok{FALSE}\NormalTok{, }\KeywordTok{length}\NormalTok{(sim_win_prob)),}
                            \KeywordTok{rep}\NormalTok{(}\OtherTok{TRUE}\NormalTok{, }\KeywordTok{length}\NormalTok{(sim_win_prob2))),}
                   \DataTypeTok{Prob_Type =} \StringTok{"Winning Probabilities"}\NormalTok{)}

\NormalTok{test3 <-}\StringTok{ }\KeywordTok{data.table}\NormalTok{(}\DataTypeTok{simdist =} \KeywordTok{c}\NormalTok{(sim_ml_prob, sim_ml_prob2),}
                   \DataTypeTok{coef =} \KeywordTok{c}\NormalTok{(}\KeywordTok{rep}\NormalTok{(}\OtherTok{FALSE}\NormalTok{, }\KeywordTok{length}\NormalTok{(sim_ml_prob)),}
                            \KeywordTok{rep}\NormalTok{(}\OtherTok{TRUE}\NormalTok{, }\KeywordTok{length}\NormalTok{(sim_ml_prob2))),}
                   \DataTypeTok{Prob_Type =} \StringTok{"Machine Learning"}\NormalTok{)}

\NormalTok{test <-}\StringTok{ }\KeywordTok{rbind}\NormalTok{(test1,test2,test3)}
\end{Highlighting}
\end{Shaded}

~

Recall - actual value is (won opportunities) is
\ensuremath{1.2784732\times 10^{8}}

~

Revenue Distributions with and without coefficient by Probability types:

\includegraphics{pvalDist_files/figure-latex/plot_dens-1.pdf}

~

Revenue Distribution Comparison with changes coefs applied:

\includegraphics{pvalDist_files/figure-latex/plot_dens_coef-1.pdf}

\hypertarget{conclusion} \&
\texttt{Winning\_Probability} probabilities catches it at the edge of
distributions which tells that these probabilities are overestimated
(too positive).

On the other hand \texttt{Predicted\ Probability} which is
\texttt{Machine\ Learning} (``hand made'') catches the actual Revenue
near the mode (the most frequent bin) which is a very good sign of these
probabilities are being accurate thou is the best choice to rely on for
simulation of Revenue Distribution relative to the other available
options.

\hypertarget{forecast-on-revenue-distribution-upon-existing-open-opportunities.}{%
\subsection{Forecast on Revenue Distribution upon existing open
Opportunities.}\label{forecast-on-revenue-distribution-upon-existing-open-opportunities.}}

\hypertarget{source-data-retrieve}{%
\subsubsection{Source Data Retrieve:}\label{source-data-retrieve}}

\begin{Shaded}
\begin{Highlighting}[]
\KeywordTok{source}\NormalTok{(}\StringTok{"../../R_OPPO_PROB/_oppoLastWeekly.R"}\NormalTok{, }\DataTypeTok{local =} \OtherTok{TRUE}\NormalTok{)}
\KeywordTok{setwd}\NormalTok{(}\StringTok{"../../R_OPPO_PROB"}\NormalTok{)}
\NormalTok{l.opps <-}\StringTok{ }\KeywordTok{oppoLastWeekly}\NormalTok{()}
\end{Highlighting}
\end{Shaded}

filtering not-closed Opportunities

\begin{Shaded}
\begin{Highlighting}[]
\NormalTok{lst.open <-}\StringTok{ }\NormalTok{l.opps[IsClosed }\OperatorTok{==}\StringTok{ }\DecValTok{0}\NormalTok{]}
\end{Highlighting}
\end{Shaded}

Leaving only IDs and probabilities:

\begin{Shaded}
\begin{Highlighting}[]
\NormalTok{lst.probs <-}\StringTok{ }\NormalTok{lst.open[, .(}\StringTok{`}\DataTypeTok{Opportunity ID}\StringTok{`}\NormalTok{, }\StringTok{`}\DataTypeTok{Probability %}\StringTok{`}\NormalTok{, }
\NormalTok{             Winning_Probability, }\StringTok{`}\DataTypeTok{Predicted Probability}\StringTok{`}\NormalTok{, }\StringTok{`}\DataTypeTok{Amount USD}\StringTok{`}\NormalTok{)]}
\end{Highlighting}
\end{Shaded}

\begin{Shaded}
\begin{Highlighting}[]
\CommentTok{# skim(lst.probs)}
\end{Highlighting}
\end{Shaded}

Replacing NAs in \texttt{Predicted\ Probability} \&
\texttt{Winning\_Probability} w/\texttt{Probability\ \%}

\begin{Shaded}
\begin{Highlighting}[]
\KeywordTok{require}\NormalTok{(dplyr)}
\NormalTok{lst.probs}\OperatorTok{$}\StringTok{`}\DataTypeTok{Predicted Probability}\StringTok{`}\NormalTok{ <-}\StringTok{ }
\StringTok{    }\KeywordTok{case_when}\NormalTok{(}
        \KeywordTok{is.na}\NormalTok{(lst.probs}\OperatorTok{$}\StringTok{`}\DataTypeTok{Predicted Probability}\StringTok{`}\NormalTok{) }\OperatorTok{~}\StringTok{ }\NormalTok{lst.probs}\OperatorTok{$}\StringTok{`}\DataTypeTok{Probability %}\StringTok{`}\NormalTok{,}
        \OtherTok{TRUE} \OperatorTok{~}\StringTok{ }\NormalTok{lst.probs}\OperatorTok{$}\StringTok{`}\DataTypeTok{Predicted Probability}\StringTok{`}\NormalTok{)}

\NormalTok{lst.probs}\OperatorTok{$}\NormalTok{Winning_Probability <-}\StringTok{ }
\StringTok{    }\KeywordTok{case_when}\NormalTok{(}
        \KeywordTok{is.na}\NormalTok{(lst.probs}\OperatorTok{$}\NormalTok{Winning_Probability) }\OperatorTok{~}\StringTok{ }\NormalTok{lst.probs}\OperatorTok{$}\StringTok{`}\DataTypeTok{Probability %}\StringTok{`}\NormalTok{,}
        \OtherTok{TRUE} \OperatorTok{~}\StringTok{ }\NormalTok{lst.probs}\OperatorTok{$}\NormalTok{Winning_Probability)}
\CommentTok{# skim(lst.probs)}
\end{Highlighting}
\end{Shaded}

Simulating future Revenue Distribution upon open Opportunaties:

\begin{Shaded}
\begin{Highlighting}[]
\NormalTok{sim_pp3 <-}\StringTok{ }\KeywordTok{pvalDist}\NormalTok{(}\DataTypeTok{pvec =}\NormalTok{ lst.probs}\OperatorTok{$}\StringTok{`}\DataTypeTok{Probability %}\StringTok{`}\NormalTok{,}
                   \DataTypeTok{valvec =}\NormalTok{ lst.probs}\OperatorTok{$}\StringTok{`}\DataTypeTok{Amount USD}\StringTok{`}\NormalTok{,}
                   \DataTypeTok{rep =} \DecValTok{10000}\NormalTok{,}
                   \DataTypeTok{parallel =} \OtherTok{TRUE}\NormalTok{,}
                   \DataTypeTok{coefs =}\NormalTok{ coefs)}
\end{Highlighting}
\end{Shaded}

\begin{verbatim}
## <pvalDist> parallel
\end{verbatim}

\begin{verbatim}
## <pMinLen>: 322 zeroes & ones are removed / total 1518 p
\end{verbatim}

\begin{verbatim}
## 0.01 minimum length for bernulli success-failure cond: 1000
\end{verbatim}

\begin{Shaded}
\begin{Highlighting}[]
\NormalTok{sim_win_prob3 <-}\StringTok{ }\KeywordTok{pvalDist}\NormalTok{(}\DataTypeTok{pvec =}\NormalTok{ lst.probs}\OperatorTok{$}\NormalTok{Winning_Probability,}
                         \DataTypeTok{valvec =}\NormalTok{ lst.probs}\OperatorTok{$}\StringTok{`}\DataTypeTok{Amount USD}\StringTok{`}\NormalTok{,}
                         \DataTypeTok{rep =} \DecValTok{10000}\NormalTok{,}
                         \DataTypeTok{parallel =} \OtherTok{TRUE}\NormalTok{,}
                         \DataTypeTok{coefs =}\NormalTok{ coefs)}
\end{Highlighting}
\end{Shaded}

\begin{verbatim}
## <pvalDist> parallel
\end{verbatim}

\begin{verbatim}
## <pMinLen>: 112 zeroes & ones are removed / total 1518 p
\end{verbatim}

\begin{verbatim}
## 0.1 minimum length for bernulli success-failure cond: 101
\end{verbatim}

\begin{Shaded}
\begin{Highlighting}[]
\NormalTok{sim_ml_prob3 <-}\StringTok{ }\KeywordTok{pvalDist}\NormalTok{(}\DataTypeTok{pvec =}\NormalTok{ lst.probs}\OperatorTok{$}\StringTok{`}\DataTypeTok{Predicted Probability}\StringTok{`}\NormalTok{,}
                         \DataTypeTok{valvec =}\NormalTok{ lst.probs}\OperatorTok{$}\StringTok{`}\DataTypeTok{Amount USD}\StringTok{`}\NormalTok{,}
                         \DataTypeTok{rep =} \DecValTok{10000}\NormalTok{,}
                         \DataTypeTok{parallel =} \OtherTok{TRUE}\NormalTok{,}
                         \DataTypeTok{coefs =}\NormalTok{ coefs)}
\end{Highlighting}
\end{Shaded}

\begin{verbatim}
## <pvalDist> parallel
\end{verbatim}

\begin{verbatim}
## 0.0014 minimum length for bernulli success-failure cond: 7143
\end{verbatim}

Gathering simulations on different probabilities into single dataset

\begin{Shaded}
\begin{Highlighting}[]
\NormalTok{dtFromSims <-}\StringTok{ }\ControlFlowTok{function}\NormalTok{(vecs_list, names) \{}
        \KeywordTok{require}\NormalTok{(tidyr)}
        \KeywordTok{stopifnot}\NormalTok{(}\KeywordTok{length}\NormalTok{(vecs_list) }\OperatorTok{==}\StringTok{ }\KeywordTok{length}\NormalTok{(names))}
        \KeywordTok{names}\NormalTok{(vecs_list) <-}\StringTok{ }\NormalTok{names}
        \KeywordTok{return}\NormalTok{(}\KeywordTok{gather}\NormalTok{(}\KeywordTok{as.data.table}\NormalTok{(vecs_list)))}
\NormalTok{\}}

\NormalTok{f.sim <-}\StringTok{ }\KeywordTok{dtFromSims}\NormalTok{(}\KeywordTok{list}\NormalTok{(sim_pp3, sim_win_prob3, sim_ml_prob3),}
                    \KeywordTok{c}\NormalTok{(}\StringTok{"Probability %"}\NormalTok{, }\StringTok{"Winning_Probability"}\NormalTok{,}
                      \StringTok{"Predicted Probability"}\NormalTok{))}
\end{Highlighting}
\end{Shaded}

\begin{verbatim}
## Loading required package: tidyr
\end{verbatim}

Ploting Future Revenue Distribution Simulation:

\begin{Shaded}
\begin{Highlighting}[]
\KeywordTok{densIdentPlot}\NormalTok{(f.sim, }\DataTypeTok{x =} \StringTok{"value"}\NormalTok{, }\DataTypeTok{fill =} \StringTok{"key"}\NormalTok{, }\DataTypeTok{suffix =} \StringTok{"M"}\NormalTok{)}
\end{Highlighting}
\end{Shaded}

\includegraphics{pvalDist_files/figure-latex/unnamed-chunk-45-1.pdf}

\hypertarget{distribution-comparison-for-opportunities-on-2019-04-01}{%
\section{Distribution comparison for Opportunities on
2019-04-01:}\label{distribution-comparison-for-opportunities-on-2019-04-01}}

\begin{verbatim}
## <pvalDist> parallel
\end{verbatim}

\begin{verbatim}
## <pMinLen>: 194 zeroes & ones are removed / total 907 p
\end{verbatim}

\begin{verbatim}
## 0.01 minimum length for bernulli success-failure cond: 1000
\end{verbatim}

\begin{verbatim}
## <pvalDist> parallel
\end{verbatim}

\begin{verbatim}
## <pMinLen>: 132 zeroes & ones are removed / total 907 p
\end{verbatim}

\begin{verbatim}
## 0.01 minimum length for bernulli success-failure cond: 1000
\end{verbatim}

\begin{verbatim}
## <pvalDist> parallel
\end{verbatim}

\begin{verbatim}
## <pMinLen>: 2 zeroes & ones are removed / total 907 p
\end{verbatim}

\begin{verbatim}
## 0.00370943534395735 minimum length for bernulli success-failure cond: 2696
\end{verbatim}

\begin{tabular}{l|l|l|l|r|l}
\hline
variable & type & stat & level & value & formatted\\
\hline
sim\_941\_m & numeric & missing & .all & 0 & 0\\
\hline
sim\_941\_m & numeric & complete & .all & 10000 & 10000\\
\hline
sim\_941\_m & numeric & n & .all & 10000 & 10000\\
\hline
sim\_941\_m & numeric & mean & .all & 119193961 & 1.2e+08\\
\hline
sim\_941\_m & numeric & sd & .all & 9702133 & 9702133.42\\
\hline
sim\_941\_m & numeric & p0 & .all & 90781083 & 9.1e+07\\
\hline
sim\_941\_m & numeric & p25 & .all & 112462507 & 1.1e+08\\
\hline
sim\_941\_m & numeric & p50 & .all & 118691202 & 1.2e+08\\
\hline
sim\_941\_m & numeric & p75 & .all & 125370796 & 1.3e+08\\
\hline
sim\_941\_m & numeric & p100 & .all & 166457084 & 1.7e+08\\
\hline
sim\_941\_m & numeric & hist & .all & NA & ▁▃▇▇▃▁▁▁\\
\hline
\end{tabular}

\hypertarget{actual-value}{%
\subsection{Actual Value}\label{actual-value}}

\begin{Shaded}
\begin{Highlighting}[]
\KeywordTok{print}\NormalTok{(act.rev190401 <-}\StringTok{ }\KeywordTok{sum}\NormalTok{(d2019_}\DecValTok{04}\NormalTok{_01sl[now_IsWon }\OperatorTok{==}\StringTok{ }\DecValTok{1}\NormalTok{]}\OperatorTok{$}\NormalTok{now_AmountUSD))}
\end{Highlighting}
\end{Shaded}

\begin{verbatim}
## [1] 118407350
\end{verbatim}

\begin{Shaded}
\begin{Highlighting}[]
\CommentTok{# skim(d2019_04_01sl) %>% pander()}
\end{Highlighting}
\end{Shaded}

\hypertarget{plot-2019-04-01-distributions.}{%
\subsection{Plot 2019-04-01
distributions.}\label{plot-2019-04-01-distributions.}}

\includegraphics{pvalDist_files/figure-latex/unnamed-chunk-48-1.pdf}

\hypertarget{simulating-finlob-only-from-2019-04-01}{%
\section{SIMULATING FINLOB ONLY FROM
2019-04-01}\label{simulating-finlob-only-from-2019-04-01}}

\begin{Shaded}
\begin{Highlighting}[]
\NormalTok{d190401_finlob <-}\StringTok{ }\NormalTok{d2019_}\DecValTok{04}\NormalTok{_}\DecValTok{01}\NormalTok{[}\StringTok{`}\DataTypeTok{Opportunity LOB}\StringTok{`} \OperatorTok{==}\StringTok{ "FIN LOB"}\NormalTok{]}
\NormalTok{d190401_finlob <-}\StringTok{ }\NormalTok{d190401_finlob[, }\KeywordTok{c}\NormalTok{(}\StringTok{"Opportunity_ID"}\NormalTok{,}
                                     \StringTok{"Probability %"}\NormalTok{,}
                                     \StringTok{"Winning_Probability"}\NormalTok{,}
                                     \StringTok{"Probability_Of_Success"}\NormalTok{,}
                                     \StringTok{"Predicted Probability"}\NormalTok{,}
                                     \StringTok{"now_IsWon"}\NormalTok{,}
                                     \StringTok{"Amount USD"}\NormalTok{,}
                                     \StringTok{"now_AmountUSD"}\NormalTok{)]}

\KeywordTok{print}\NormalTok{(act.revfinlob190401 <-}\StringTok{ }\KeywordTok{sum}\NormalTok{(d190401_finlob[now_IsWon }\OperatorTok{==}\StringTok{ }\DecValTok{1}\NormalTok{]}\OperatorTok{$}\NormalTok{now_AmountUSD))}
\end{Highlighting}
\end{Shaded}

\begin{verbatim}
## [1] 44704174
\end{verbatim}

\begin{Shaded}
\begin{Highlighting}[]
\NormalTok{sim_}\DecValTok{941}\NormalTok{_finlob_p <-}\StringTok{ }\KeywordTok{pvalDist}\NormalTok{(}\DataTypeTok{pvec =}\NormalTok{ d190401_finlob}\OperatorTok{$}\StringTok{`}\DataTypeTok{Probability %}\StringTok{`}\NormalTok{,}
                      \DataTypeTok{valvec =}\NormalTok{ d190401_finlob}\OperatorTok{$}\StringTok{`}\DataTypeTok{Amount USD}\StringTok{`}\NormalTok{,}
                      \DataTypeTok{rep =} \DecValTok{10000}\NormalTok{,}
                      \DataTypeTok{coefs =}\NormalTok{ coefs,}
                      \DataTypeTok{parallel =} \OtherTok{TRUE}\NormalTok{)}
\end{Highlighting}
\end{Shaded}

\begin{verbatim}
## <pvalDist> parallel
\end{verbatim}

\begin{verbatim}
## <pMinLen>: 148 zeroes & ones are removed / total 454 p
\end{verbatim}

\begin{verbatim}
## 0.01 minimum length for bernulli success-failure cond: 1000
\end{verbatim}

\begin{Shaded}
\begin{Highlighting}[]
\NormalTok{sim_}\DecValTok{941}\NormalTok{_finlob_w <-}\StringTok{ }\KeywordTok{pvalDist}\NormalTok{(}\DataTypeTok{pvec =}\NormalTok{ d190401_finlob}\OperatorTok{$}\NormalTok{Winning_Probability,}
                      \DataTypeTok{valvec =}\NormalTok{ d190401_finlob}\OperatorTok{$}\StringTok{`}\DataTypeTok{Amount USD}\StringTok{`}\NormalTok{,}
                      \DataTypeTok{rep =} \DecValTok{10000}\NormalTok{,}
                      \DataTypeTok{coefs =}\NormalTok{ coefs,}
                      \DataTypeTok{parallel =} \OtherTok{TRUE}\NormalTok{)}
\end{Highlighting}
\end{Shaded}

\begin{verbatim}
## <pvalDist> parallel
\end{verbatim}

\begin{verbatim}
## <pMinLen>: 100 zeroes & ones are removed / total 454 p
\end{verbatim}

\begin{verbatim}
## 0.01 minimum length for bernulli success-failure cond: 1000
\end{verbatim}

\begin{Shaded}
\begin{Highlighting}[]
\NormalTok{sim_}\DecValTok{941}\NormalTok{_finlob_m <-}\StringTok{ }\KeywordTok{pvalDist}\NormalTok{(}\DataTypeTok{pvec =}\NormalTok{ d190401_finlob}\OperatorTok{$}\StringTok{`}\DataTypeTok{Predicted Probability}\StringTok{`}\NormalTok{,}
                      \DataTypeTok{valvec =}\NormalTok{ d190401_finlob}\OperatorTok{$}\StringTok{`}\DataTypeTok{Amount USD}\StringTok{`}\NormalTok{,}
                      \DataTypeTok{rep =} \DecValTok{10000}\NormalTok{,}
                      \DataTypeTok{coefs =}\NormalTok{ coefs,}
                      \DataTypeTok{parallel =} \OtherTok{TRUE}\NormalTok{)}
\end{Highlighting}
\end{Shaded}

\begin{verbatim}
## <pvalDist> parallel
\end{verbatim}

\begin{verbatim}
## <pMinLen>: 2 zeroes & ones are removed / total 454 p
\end{verbatim}

\begin{verbatim}
## 0.00380393919010723 minimum length for bernulli success-failure cond: 2629
\end{verbatim}

\begin{Shaded}
\begin{Highlighting}[]
\NormalTok{f_sim190401_finlob <-}\StringTok{ }\KeywordTok{dtFromSims}\NormalTok{(}\KeywordTok{list}\NormalTok{(sim_}\DecValTok{941}\NormalTok{_finlob_p,}
\NormalTok{                                      sim_}\DecValTok{941}\NormalTok{_finlob_w,}
\NormalTok{                                      sim_}\DecValTok{941}\NormalTok{_finlob_m),}
                    \KeywordTok{c}\NormalTok{(}\StringTok{"Probability %"}\NormalTok{, }\StringTok{"Winning_Probability"}\NormalTok{,}
                      \StringTok{"Predicted Probability"}\NormalTok{))}

\KeywordTok{densIdentPlot}\NormalTok{(f_sim190401_finlob, }\DataTypeTok{x =} \StringTok{"value"}\NormalTok{, }\DataTypeTok{fill =} \StringTok{"key"}\NormalTok{, }\DataTypeTok{suffix =} \StringTok{"M"}\NormalTok{) }\OperatorTok{+}
\StringTok{    }\KeywordTok{geom_text}\NormalTok{(}\KeywordTok{aes}\NormalTok{(}\DataTypeTok{x =}\NormalTok{ act.revfinlob190401, }\DataTypeTok{y =} \DecValTok{0}\NormalTok{),}
    \DataTypeTok{label =} \StringTok{"Actual Revenue"}\NormalTok{,}
    \DataTypeTok{color =} \StringTok{"black"}\NormalTok{,}
    \DataTypeTok{size =} \DecValTok{4}\NormalTok{,}
    \DataTypeTok{angle =} \DecValTok{90}\NormalTok{,}
    \DataTypeTok{vjust =} \FloatTok{1.5}\NormalTok{,}
    \DataTypeTok{hjust =} \FloatTok{-0.5}\NormalTok{) }\OperatorTok{+}
\StringTok{  }\KeywordTok{geom_vline}\NormalTok{(}\DataTypeTok{xintercept =}\NormalTok{ act.revfinlob190401, }\DataTypeTok{linetype =} \StringTok{"solid"}\NormalTok{, }\DataTypeTok{size =} \DecValTok{1}\NormalTok{, }\DataTypeTok{color =} \StringTok{"black"}\NormalTok{) }\OperatorTok{+}
\StringTok{  }\KeywordTok{labs}\NormalTok{(}\DataTypeTok{caption =} \KeywordTok{paste0}\NormalTok{(}\StringTok{"Actual Revenue: "}\NormalTok{, }\KeywordTok{dollar}\NormalTok{(}\KeywordTok{round}\NormalTok{(act.revfinlob190401))))}
\end{Highlighting}
\end{Shaded}

\includegraphics{pvalDist_files/figure-latex/finlob_sim_941-1.pdf}

\begin{Shaded}
\begin{Highlighting}[]
\CommentTok{# names(d2019_04_01)}
\CommentTok{# d190401_finlob}
\end{Highlighting}
\end{Shaded}

\hypertarget{simulating-particular-client.-say-daimler.}{%
\section{Simulating particular client. Say
Daimler.}\label{simulating-particular-client.-say-daimler.}}

\begin{Shaded}
\begin{Highlighting}[]
\NormalTok{daimler}\FloatTok{.190401}\NormalTok{ <-}\StringTok{ }\NormalTok{d2019_}\DecValTok{04}\NormalTok{_}\DecValTok{01}\NormalTok{[}\StringTok{`}\DataTypeTok{Account Name}\StringTok{`} \OperatorTok{==}\StringTok{ "Daimler"}\NormalTok{]}

\NormalTok{daimler}\FloatTok{.190401}\NormalTok{ <-}\StringTok{ }\NormalTok{daimler}\FloatTok{.190401}\NormalTok{[, }\KeywordTok{c}\NormalTok{(}\StringTok{"Opportunity_ID"}\NormalTok{,}
                                     \StringTok{"Probability %"}\NormalTok{,}
                                     \StringTok{"Winning_Probability"}\NormalTok{,}
                                     \StringTok{"Probability_Of_Success"}\NormalTok{,}
                                     \StringTok{"Predicted Probability"}\NormalTok{,}
                                     \StringTok{"now_IsWon"}\NormalTok{,}
                                     \StringTok{"Amount USD"}\NormalTok{,}
                                     \StringTok{"now_AmountUSD"}\NormalTok{)]}

\KeywordTok{print}\NormalTok{(actrev_daimler}\FloatTok{.190401}\NormalTok{ <-}\StringTok{ }\KeywordTok{sum}\NormalTok{(daimler}\FloatTok{.190401}\NormalTok{[now_IsWon }\OperatorTok{==}\StringTok{ }\DecValTok{1}\NormalTok{]}\OperatorTok{$}\NormalTok{now_AmountUSD))}
\end{Highlighting}
\end{Shaded}

\begin{verbatim}
## [1] 2099575
\end{verbatim}

\begin{Shaded}
\begin{Highlighting}[]
\NormalTok{sim_}\DecValTok{941}\NormalTok{_daimler_p <-}\StringTok{ }\KeywordTok{pvalDist}\NormalTok{(}\DataTypeTok{pvec =}\NormalTok{ daimler}\FloatTok{.190401}\OperatorTok{$}\StringTok{`}\DataTypeTok{Probability %}\StringTok{`}\NormalTok{,}
                      \DataTypeTok{valvec =}\NormalTok{ daimler}\FloatTok{.190401}\OperatorTok{$}\StringTok{`}\DataTypeTok{Amount USD}\StringTok{`}\NormalTok{,}
                      \DataTypeTok{rep =} \DecValTok{10000}\NormalTok{,}
                      \DataTypeTok{coefs =}\NormalTok{ coefs,}
                      \DataTypeTok{parallel =} \OtherTok{TRUE}\NormalTok{)}
\end{Highlighting}
\end{Shaded}

\begin{verbatim}
## <pvalDist> parallel
\end{verbatim}

\begin{verbatim}
## 0.01 minimum length for bernulli success-failure cond: 1000
\end{verbatim}

\begin{Shaded}
\begin{Highlighting}[]
\NormalTok{sim_}\DecValTok{941}\NormalTok{_daimler_w <-}\StringTok{ }\KeywordTok{pvalDist}\NormalTok{(}\DataTypeTok{pvec =}\NormalTok{ daimler}\FloatTok{.190401}\OperatorTok{$}\NormalTok{Winning_Probability,}
                      \DataTypeTok{valvec =}\NormalTok{ daimler}\FloatTok{.190401}\OperatorTok{$}\StringTok{`}\DataTypeTok{Amount USD}\StringTok{`}\NormalTok{,}
                      \DataTypeTok{rep =} \DecValTok{10000}\NormalTok{,}
                      \DataTypeTok{coefs =}\NormalTok{ coefs,}
                      \DataTypeTok{parallel =} \OtherTok{TRUE}\NormalTok{)}
\end{Highlighting}
\end{Shaded}

\begin{verbatim}
## <pvalDist> parallel
\end{verbatim}

\begin{verbatim}
## <pMinLen>: 1 zeroes & ones are removed / total 12 p
\end{verbatim}

\begin{verbatim}
## 0.1 minimum length for bernulli success-failure cond: 101
\end{verbatim}

\begin{Shaded}
\begin{Highlighting}[]
\NormalTok{sim_}\DecValTok{941}\NormalTok{_daimler_m <-}\StringTok{ }\KeywordTok{pvalDist}\NormalTok{(}\DataTypeTok{pvec =}\NormalTok{ daimler}\FloatTok{.190401}\OperatorTok{$}\StringTok{`}\DataTypeTok{Predicted Probability}\StringTok{`}\NormalTok{,}
                      \DataTypeTok{valvec =}\NormalTok{ daimler}\FloatTok{.190401}\OperatorTok{$}\StringTok{`}\DataTypeTok{Amount USD}\StringTok{`}\NormalTok{,}
                      \DataTypeTok{rep =} \DecValTok{10000}\NormalTok{,}
                      \DataTypeTok{coefs =}\NormalTok{ coefs,}
                      \DataTypeTok{parallel =} \OtherTok{TRUE}\NormalTok{)}
\end{Highlighting}
\end{Shaded}

\begin{verbatim}
## <pvalDist> parallel
\end{verbatim}

\begin{verbatim}
## 0.0207382723993602 minimum length for bernulli success-failure cond: 483
\end{verbatim}

\begin{Shaded}
\begin{Highlighting}[]
\NormalTok{f_sim190401_daimler <-}\StringTok{ }\KeywordTok{dtFromSims}\NormalTok{(}\KeywordTok{list}\NormalTok{(sim_}\DecValTok{941}\NormalTok{_daimler_p,}
\NormalTok{                                      sim_}\DecValTok{941}\NormalTok{_daimler_w,}
\NormalTok{                                      sim_}\DecValTok{941}\NormalTok{_daimler_m),}
                    \KeywordTok{c}\NormalTok{(}\StringTok{"Probability %"}\NormalTok{, }\StringTok{"Winning_Probability"}\NormalTok{,}
                      \StringTok{"Predicted Probability"}\NormalTok{))}

\KeywordTok{densIdentPlot}\NormalTok{(f_sim190401_daimler, }\DataTypeTok{x =} \StringTok{"value"}\NormalTok{, }\DataTypeTok{fill =} \StringTok{"key"}\NormalTok{, }\DataTypeTok{suffix =} \StringTok{"M"}\NormalTok{) }\OperatorTok{+}
\StringTok{    }\KeywordTok{geom_text}\NormalTok{(}\KeywordTok{aes}\NormalTok{(}\DataTypeTok{x =}\NormalTok{ actrev_daimler}\FloatTok{.190401}\NormalTok{, }\DataTypeTok{y =} \DecValTok{0}\NormalTok{),}
    \DataTypeTok{label =} \StringTok{"Actual Revenue"}\NormalTok{,}
    \DataTypeTok{color =} \StringTok{"black"}\NormalTok{,}
    \DataTypeTok{size =} \DecValTok{4}\NormalTok{,}
    \DataTypeTok{angle =} \DecValTok{90}\NormalTok{,}
    \DataTypeTok{vjust =} \FloatTok{1.5}\NormalTok{,}
    \DataTypeTok{hjust =} \FloatTok{-0.5}\NormalTok{) }\OperatorTok{+}
\StringTok{  }\KeywordTok{geom_vline}\NormalTok{(}\DataTypeTok{xintercept =}\NormalTok{ actrev_daimler}\FloatTok{.190401}\NormalTok{, }\DataTypeTok{linetype =} \StringTok{"solid"}\NormalTok{, }\DataTypeTok{size =} \DecValTok{1}\NormalTok{, }\DataTypeTok{color =} \StringTok{"black"}\NormalTok{) }\OperatorTok{+}
\StringTok{  }\KeywordTok{labs}\NormalTok{(}\DataTypeTok{caption =} \KeywordTok{paste0}\NormalTok{(}\StringTok{"Actual Revenue: "}\NormalTok{, }\KeywordTok{dollar}\NormalTok{(}\KeywordTok{round}\NormalTok{(actrev_daimler}\FloatTok{.190401}\NormalTok{))))}
\end{Highlighting}
\end{Shaded}

\includegraphics{pvalDist_files/figure-latex/sim_daimler-1.pdf}

\begin{Shaded}
\begin{Highlighting}[]
\CommentTok{# names(d2019_04_01)}
\CommentTok{# d190401_finlob}

\CommentTok{# sort(unique(d2019_04_01$`Account Name`))}
\end{Highlighting}
\end{Shaded}

\begin{Shaded}
\begin{Highlighting}[]
\KeywordTok{source}\NormalTok{(}\StringTok{"../../R_OPPO_PROB/_oppoProb.R"}\NormalTok{)}
\KeywordTok{setwd}\NormalTok{(}\StringTok{"../../R_OPPO_PROB"}\NormalTok{)}
\NormalTok{ml_}\DecValTok{941}\NormalTok{ <-}\StringTok{ }\KeywordTok{oppoProb}\NormalTok{(}\DataTypeTok{given_date =} \StringTok{"2019-04-01"}\NormalTok{)}
\end{Highlighting}
\end{Shaded}

\begin{verbatim}
## <oppoProb> Execute
\end{verbatim}

\begin{verbatim}
## <wkAgg> Feature Engineering
\end{verbatim}

\begin{verbatim}
## <wkAgg> excluding observation above 2019-04-01
\end{verbatim}

\begin{verbatim}
## <wkAgg> 8708 OpIDs removed as all stages cyrillic
\end{verbatim}

\begin{verbatim}
## <compTrainData> Opportunity Probabilities as for 2019-04-01
\end{verbatim}

\begin{verbatim}
## <compTrainData> 0 removed due to upper limit date
\end{verbatim}

\begin{verbatim}
## <compTrainData> 13014 rows removed due to bottom limit date
\end{verbatim}

\begin{verbatim}
## <compTrainData> 0 rows removed as rm.badstart
\end{verbatim}

\begin{Shaded}
\begin{Highlighting}[]
\NormalTok{d2019_}\DecValTok{04}\NormalTok{_01sl[, }\StringTok{`}\DataTypeTok{Predicted Probability}\StringTok{`} \OperatorTok{:}\ErrorTok{=}\StringTok{ }\OtherTok{NULL}\NormalTok{]}
\NormalTok{lj_941c <-}\StringTok{ }\KeywordTok{setDT}\NormalTok{(}\KeywordTok{left_join}\NormalTok{(d2019_}\DecValTok{04}\NormalTok{_01sl, ml_}\DecValTok{941}\NormalTok{[, }\KeywordTok{c}\NormalTok{(}\StringTok{"Opportunity_ID"}\NormalTok{, }\StringTok{"Probability"}\NormalTok{)], }\DataTypeTok{by =} \KeywordTok{c}\NormalTok{(}\StringTok{"Opportunity_ID"}\NormalTok{ =}\StringTok{ "Opportunity_ID"}\NormalTok{)))}
\end{Highlighting}
\end{Shaded}

\begin{verbatim}
## Warning: Column `Opportunity_ID` joining character vector and factor,
## coercing into character vector
\end{verbatim}

\begin{Shaded}
\begin{Highlighting}[]
\NormalTok{lj_}\DecValTok{941}\NormalTok{ <-}\StringTok{ }\KeywordTok{copy}\NormalTok{(lj_941c)}

\NormalTok{lj_}\DecValTok{941}\OperatorTok{$}\NormalTok{Probability <-}\StringTok{ }\KeywordTok{case_when}\NormalTok{(}\KeywordTok{is.na}\NormalTok{(lj_}\DecValTok{941}\OperatorTok{$}\NormalTok{Probability) }\OperatorTok{~}\StringTok{ }\NormalTok{lj_}\DecValTok{941}\OperatorTok{$}\StringTok{`}\DataTypeTok{Probability %}\StringTok{`}\NormalTok{,}
                                \OtherTok{TRUE} \OperatorTok{~}\StringTok{ }\NormalTok{lj_}\DecValTok{941}\OperatorTok{$}\NormalTok{Probability)}

\NormalTok{sim_}\DecValTok{941}\NormalTok{_m2 <-}\StringTok{ }\KeywordTok{pvalDist}\NormalTok{(}\DataTypeTok{pvec =}\NormalTok{ lj_}\DecValTok{941}\OperatorTok{$}\NormalTok{Probability,}
                       \DataTypeTok{valvec =}\NormalTok{ lj_}\DecValTok{941}\OperatorTok{$}\StringTok{`}\DataTypeTok{Amount USD}\StringTok{`}\NormalTok{,}
                       \DataTypeTok{rep =} \DecValTok{10000}\NormalTok{,}
                       \DataTypeTok{coefs =}\NormalTok{ coefs,}
                       \DataTypeTok{parallel =} \OtherTok{TRUE}\NormalTok{)}
\end{Highlighting}
\end{Shaded}

\begin{verbatim}
## <pvalDist> parallel
\end{verbatim}

\begin{verbatim}
## <pMinLen>: 2 zeroes & ones are removed / total 907 p
\end{verbatim}

\begin{verbatim}
## 0.00232899265890697 minimum length for bernulli success-failure cond: 4294
\end{verbatim}

\begin{Shaded}
\begin{Highlighting}[]
\KeywordTok{skim}\NormalTok{(sim_}\DecValTok{941}\NormalTok{_m2)}
\end{Highlighting}
\end{Shaded}

\begin{tabular}{l|l|l|l|r|l}
\hline
variable & type & stat & level & value & formatted\\
\hline
sim\_941\_m2 & numeric & missing & .all & 0 & 0\\
\hline
sim\_941\_m2 & numeric & complete & .all & 10000 & 10000\\
\hline
sim\_941\_m2 & numeric & n & .all & 10000 & 10000\\
\hline
sim\_941\_m2 & numeric & mean & .all & 110872700 & 1.1e+08\\
\hline
sim\_941\_m2 & numeric & sd & .all & 9235518 & 9235517.93\\
\hline
sim\_941\_m2 & numeric & p0 & .all & 82595036 & 8.3e+07\\
\hline
sim\_941\_m2 & numeric & p25 & .all & 104486926 & 1e+08\\
\hline
sim\_941\_m2 & numeric & p50 & .all & 110265141 & 1.1e+08\\
\hline
sim\_941\_m2 & numeric & p75 & .all & 116681528 & 1.2e+08\\
\hline
sim\_941\_m2 & numeric & p100 & .all & 154444878 & 1.5e+08\\
\hline
sim\_941\_m2 & numeric & hist & .all & NA & ▁▂▇▇▃▁▁▁\\
\hline
\end{tabular}

\hypertarget{residuals.}{%
\section{Residuals.}\label{residuals.}}

Since probability is the estimation. We need to simualte error variance.
We'll take the difference between probability \& actual outcome to have
a thougth on the error estimation.

\begin{Shaded}
\begin{Highlighting}[]
\CommentTok{# glimpse(data20181001)}
\NormalTok{d <-}\StringTok{ }\KeywordTok{copy}\NormalTok{(data20181001)}
\NormalTok{pp.res <-}\StringTok{ }\NormalTok{d}\OperatorTok{$}\StringTok{`}\DataTypeTok{Predicted Probability}\StringTok{`} \OperatorTok{-}\StringTok{ }\NormalTok{d}\OperatorTok{$}\NormalTok{now_IsWon}
\NormalTok{sf.res <-}\StringTok{ }\NormalTok{d}\OperatorTok{$}\StringTok{`}\DataTypeTok{Probability %}\StringTok{`} \OperatorTok{-}\StringTok{ }\NormalTok{d}\OperatorTok{$}\NormalTok{now_IsWon}
\NormalTok{wp.res <-}\StringTok{ }\NormalTok{d}\OperatorTok{$}\NormalTok{Winning_Probability }\OperatorTok{-}\StringTok{ }\NormalTok{d}\OperatorTok{$}\NormalTok{now_IsWon}
\KeywordTok{skim}\NormalTok{(}\KeywordTok{as.data.table}\NormalTok{(}\KeywordTok{cbind}\NormalTok{(pp.res, sf.res, wp.res)))}
\end{Highlighting}
\end{Shaded}

\begin{tabular}{l|l|l|l|r|l}
\hline
variable & type & stat & level & value & formatted\\
\hline
pp.res & numeric & missing & .all & 0.0000000 & 0\\
\hline
pp.res & numeric & complete & .all & 988.0000000 & 988\\
\hline
pp.res & numeric & n & .all & 988.0000000 & 988\\
\hline
pp.res & numeric & mean & .all & 0.0078439 & 0.0078\\
\hline
pp.res & numeric & sd & .all & 0.2991341 & 0.3\\
\hline
pp.res & numeric & p0 & .all & -0.9865802 & -0.99\\
\hline
pp.res & numeric & p25 & .all & -0.0085731 & -0.0086\\
\hline
pp.res & numeric & p50 & .all & 0.0256828 & 0.026\\
\hline
pp.res & numeric & p75 & .all & 0.0881271 & 0.088\\
\hline
pp.res & numeric & p100 & .all & 0.9786081 & 0.98\\
\hline
pp.res & numeric & hist & .all & NA & ▁▁▁▂▇▁▁▁\\
\hline
sf.res & numeric & missing & .all & 0.0000000 & 0\\
\hline
sf.res & numeric & complete & .all & 988.0000000 & 988\\
\hline
sf.res & numeric & n & .all & 988.0000000 & 988\\
\hline
sf.res & numeric & mean & .all & 0.0145142 & 0.015\\
\hline
sf.res & numeric & sd & .all & 0.3631165 & 0.36\\
\hline
sf.res & numeric & p0 & .all & -1.0000000 & -1\\
\hline
sf.res & numeric & p25 & .all & -0.0325000 & -0.032\\
\hline
sf.res & numeric & p50 & .all & 0.0000000 & 0\\
\hline
sf.res & numeric & p75 & .all & 0.2500000 & 0.25\\
\hline
sf.res & numeric & p100 & .all & 0.9000000 & 0.9\\
\hline
sf.res & numeric & hist & .all & NA & ▁▁▁▂▇▂▂▁\\
\hline
wp.res & numeric & missing & .all & 0.0000000 & 0\\
\hline
wp.res & numeric & complete & .all & 988.0000000 & 988\\
\hline
wp.res & numeric & n & .all & 988.0000000 & 988\\
\hline
wp.res & numeric & mean & .all & 0.0466599 & 0.047\\
\hline
wp.res & numeric & sd & .all & 0.3657361 & 0.37\\
\hline
wp.res & numeric & p0 & .all & -1.0000000 & -1\\
\hline
wp.res & numeric & p25 & .all & 0.0000000 & 0\\
\hline
wp.res & numeric & p50 & .all & 0.1000000 & 0.1\\
\hline
wp.res & numeric & p75 & .all & 0.2500000 & 0.25\\
\hline
wp.res & numeric & p100 & .all & 1.0000000 & 1\\
\hline
wp.res & numeric & hist & .all & NA & ▁▂▁▇▇▅▁▁\\
\hline
\end{tabular}

Random apply this error coefficient to probabilities.

\begin{Shaded}
\begin{Highlighting}[]
\NormalTok{applyErr <-}\StringTok{ }\ControlFlowTok{function}\NormalTok{(p, err) \{}
  
        \KeywordTok{require}\NormalTok{(dplyr)}
        \KeywordTok{stopifnot}\NormalTok{(}\KeywordTok{length}\NormalTok{(p) }\OperatorTok{==}\StringTok{ }\NormalTok{1L)}
\NormalTok{        xp <-}\StringTok{ }\NormalTok{p }\OperatorTok{+}\StringTok{ }\NormalTok{(p }\OperatorTok{*}\StringTok{ }\KeywordTok{sample}\NormalTok{(err, }\DataTypeTok{size =} \KeywordTok{length}\NormalTok{(p), }\DataTypeTok{replace =} \OtherTok{TRUE}\NormalTok{))}
\NormalTok{        xp <-}\StringTok{ }\KeywordTok{case_when}\NormalTok{(xp }\OperatorTok{<}\StringTok{ }\NormalTok{0L }\OperatorTok{~}\StringTok{ }\DecValTok{0}\NormalTok{,}
\NormalTok{                        xp }\OperatorTok{>}\StringTok{ }\NormalTok{1L }\OperatorTok{~}\StringTok{ }\DecValTok{1}\NormalTok{,}
                        \OtherTok{TRUE} \OperatorTok{~}\StringTok{ }\NormalTok{xp)}
        \KeywordTok{return}\NormalTok{(xp)}
  
\NormalTok{\}}

\NormalTok{sim_err_pp <-}\StringTok{ }\KeywordTok{pvalDist}\NormalTok{(}\DataTypeTok{pvec     =}\NormalTok{ lj_}\DecValTok{941}\OperatorTok{$}\NormalTok{Probability,}
                       \DataTypeTok{valvec   =}\NormalTok{ lj_}\DecValTok{941}\OperatorTok{$}\StringTok{`}\DataTypeTok{Amount USD}\StringTok{`}\NormalTok{,}
                       \DataTypeTok{rep      =} \DecValTok{10000}\NormalTok{,}
                       \DataTypeTok{coefs    =}\NormalTok{ coefs,}
                       \DataTypeTok{parallel =} \OtherTok{TRUE}\NormalTok{)}
\end{Highlighting}
\end{Shaded}

\begin{verbatim}
## <pvalDist> parallel
\end{verbatim}

\begin{verbatim}
## <pMinLen>: 2 zeroes & ones are removed / total 907 p
\end{verbatim}

\begin{verbatim}
## 0.00232899265890697 minimum length for bernulli success-failure cond: 4294
\end{verbatim}


\end{document}
